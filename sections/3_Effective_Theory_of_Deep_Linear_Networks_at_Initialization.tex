这一章主要从线性神经网络进行推导,相当于是激活函数为线性函数的MLP网络。

\subsection{Deep Linear Networks}
\label{sec:3-1}

线性神经网络中,$l+1$层第$i$个原子与$l$层原子的迭代关系如下:
\begin{equation}
    z_{i;\alpha}^{(l+1)} = b_i^{(l+1)} + \sum_{j=1}^{n_l} W_{ij}^{l+1}z_{j;\alpha}^{(l)},
\end{equation}
这里$\alpha$用来表示不同的输入样本index。实际上可以直接从输入$x$到第$l$层进行合并,变成一个矩阵相乘的形式:
\begin{equation}
    z_{i;\alpha}^{(l)} = \sum_{j_0=1}^{n_0}\sum_{j_1=1}^{n_1} \cdots \sum_{j_{l-1}=1}^{n_{l-1}}
                W_{ij_{l-1}}^{(l)} W_{j_{l-1}j_{l-2}}^{(l-1)} \cdots W_{j_1j_0}^{(1)} x_{j_0;\alpha}
              = \sum_{j=1}^{n_0} \mathcal{W}_{ij}^{(l)} x_{j;\alpha}.
\end{equation}

Here $\mathcal{W}$ is an $n_l$-by-$n_0$ matrix:
\begin{equation}
    \mathcal{W}_{ij}^{(l)} = \sum_{j_0=1}^{n_0}\sum_{j_1=1}^{n_1} \cdots \sum_{j_{l-1}=1}^{n_{l-1}}
                W_{ij_{l-1}}^{(l)} W_{j_{l-1}j_{l-2}}^{(l-1)} \cdots W_{j_1j}^{(1)}.
\end{equation}

设定每一层权重的协方差是固定的($C_W^{(l)} = C_W$)且为零均值,参数之间相互独立:
\begin{equation}
    \mathbb{E}[W_{ij}^{(l)} = 0], ~~ \mathbb{E}[W_{i_1j_1}^{(l)}W_{i_2j_2}^{(l)}] 
                                    = \delta_{i_1i_2}\delta_{j_1j_2}\frac{C_W}{n_{l-1}}
    \label{eq:expectation}
\end{equation}
这里$1/n_l$为对各层原子数进行归一化的系数,$\delta_{ij} = 1 ~ if ~ i=j, else ~ \delta_{ij} = 0$.

要研究$z^{(l)}$的分布问题,按照高斯分布的研究方式的话,主要关注$z^{(l)}$的均值和方差。均值推导如下:
\begin{equation}
    \begin{aligned}
    \mathbb{E}[z_{i;\alpha}^{(l)}] =& \mathbb{E}[\sum_{j=1}^{n_0} \mathcal{W}_{ij}^{(l)} x_{j;\alpha}] \\
     =& \sum_{j_0=1}^{n_0}\sum_{j_1=1}^{n_1} \cdots \sum_{j_{l-1}=1}^{n_{l-1}}
       \mathbb{E}[W_{ij_{l-1}}^{(l)}] \mathbb{E}[W_{j_{l-1}j_{l-2}}^{(l-1)}]
       \cdots \mathbb{E}[W_{j_1j_0}^{(1)}] x_{j_0;\alpha} = 0
    \end{aligned}
\end{equation}
这里用到了层与层之间参数相互独立的假设。接下来就是要研究$z^{(l)}$的协方差分布的问题。

\subsection{Criticality [物理学临界状态]}
\label{sec:3-2}

\subsubsection{数学推导}
这一节首先研究一下两个原子之间的协方差$\mathbb{E}[z_{i_1;\alpha_1}^{(l)}z_{i_2;\alpha_2}^{(l)}]$,
因为协方差是否为$0$决定了各原子分布是否为高斯分布(相互独立的服从高斯分布的变量的和也为高斯分布,如果不独立,分布将变得难以分析)。

\begin{equation*}
    \begin{aligned}
   \mathbb{E} [z_{i_1;\alpha_1}^{(1)}z_{i_2;\alpha_2}^{(1)}] =& \sum_{j_1,j_2=1}^{n_0} 
            \mathbb{E} [W_{i_1j_1}^{(1)}x_{j_1;\alpha_1}W_{i_2j_2}^{(1)}x_{j_2;\alpha_2}] \\
    =& \sum_{j_1,j_2=1}^{n_0} \mathbb{E} [W_{i_1j_1}^{(1)}W_{i_2j_2}^{(1)}]x_{j_1;\alpha_1}x_{j_2;\alpha_2},
    \end{aligned}
\end{equation*}
代入公式 \ref{eq:expectation},然后把第二个$\delta_{j_1j_2}$合并一下, 可以简化为:
\begin{equation*}
             = \sum_{j_1,j_2=1}^{n_0} \frac{C_W}{n_0} 
             \delta_{i_1i_2}\delta_{j_1j_2}x_{j_1;\alpha_1}x_{j_2;\alpha_2}
             = \delta_{i_1i_2}C_W \frac{1}{n_0}\sum_{j=1}^{n_0}x_{j;\alpha_1}x_{j;\alpha_2}.
\end{equation*}
这里稍微用到了维克定理的做法。
类似地,可以得到从$l$到第$l+1$的递推公式:
\begin{equation}
    \mathbb{E}[z_{i_1;\alpha_1}^{(l+1)} z_{i_2;\alpha_2}^{(l+1)}] = 
        \delta_{i_1i_2}C_W \frac{1}{n_l}\sum_{j=1}^{n_l}\mathbb{E} 
            [z_{j;\alpha_1}^{(l)}z_{j;\alpha_2}^{(l)}].
    \label{eq:l-2-lp1}
\end{equation}
对任意两个由输入维度归一化之后的内积定义如下:
\begin{equation*}
    G_{\alpha_1\alpha_2}^{(0)} \equiv \frac{1}{n_0} \sum_{i=1}^{n_0} x_{i;\alpha_1}x_{i;\alpha_2}, ~~
    G_{\alpha_1\alpha_2}^{(l)} \equiv \frac{1}{n_l} \sum_{j=1}^{n_l} 
        \mathbb{E}[z_{j;\alpha_1}^{(l)}z_{j;\alpha_2}^{(l)}],
    \label{eq:2corr-def}
\end{equation*}
根据这个定义以及公式\ref{eq:l-2-lp1},可以把从$0$到$1$以及从$l$到$l+1$递归地写成:
\begin{equation}
    \mathbb{E}[z_{i_1;\alpha_1}^{(1)}z_{i_2;\alpha_2}^{(1)}] = \delta_{i_1i_2}C_W G_{\alpha_1\alpha_2}^{(0)},
\end{equation}
\begin{equation}
    G_{\alpha_1\alpha_2}^{(l+1)} = 
        \frac{1}{n_{l+1}} \sum_{j=1}^{n_{l+1}} \mathbb{E}[z_{j;\alpha_1}^{(l+1)}z_{j;\alpha_2}^{(l+1)}] 
        = \frac{1}{n_{l+1}}\sum_{j=1}^{n_{l+1}} \delta_{jj}C_W G_{\alpha_1\alpha_2}^{(l)}
        = C_W G_{\alpha_1\alpha_2}^{(l)},
\end{equation}
由此,可以得到由两个输入$(x_{\alpha_1}, x_{\alpha_2})$协方差$G_{\alpha_1\alpha_2}^{(0)}$传递到第
$l$层的幅度为:
\begin{equation}
    G_{\alpha_1\alpha_2}^{(l)} = (C_W)^l G_{\alpha_1\alpha_2}^{(0)}.
    \label{eq:recursion}
\end{equation} 

\subsubsection{物理学解释}
\label{sec:2point-ana}
根据迭代公式 \ref{eq:recursion}, 当$C_W > 1$, 随着层数$l$递增,$G_{\alpha_1\alpha_2} \to \infty$ (exploding,
numerical instability); 当 $C_W < 1$, $G_{\alpha_1\alpha_2} \to 0$ (valishing, loss of information), 
原文中称之为``trivial fixed point''. 这里是从协方差传递的角度进行分析, 
\emph{神经网络初始化策略研究中,有类似的结论,通过研究单个原子的均值和方差随层数增加的变化,提出了xavier初始化策略}.
(\xu{action: 附上reference}). 当然,第$L$层的原子的方差/幅度也可以通过如下公式得到(均值为$0$):
\begin{equation*}
    G_{\alpha\alpha}^L = \mathbb{E} [\frac{1}{n_L} \sum_{j=1}^{n_L}(z_{j;\alpha}^{(L)})^2].
\end{equation*}
接上文,当$C_W=1$的时候 (Criticality, 临界状态),各层的协方差都是固定的 
$G_{\alpha_1\alpha_2}^{(l)} = G_{\alpha_1\alpha_2}^{(0)} \equiv G_{\alpha_1\alpha_2}^*$. 这里
协方差保持一致的意义在于,输入数据的结构信息(structure)在网络传递的过程中得到了保留。

\subsection{Fluctuations [物理学中的扰动]}
\label{sec:3-3}
\subsubsection{数学推导}
上一节研究了$2$个原子之间的协方差关系,为了进行扩展,接下来研究$4$个原子的协方差关系($4$-point correlator),
基本思路是拆分为$2$个原子之间协方差的组合。
跟公式 \ref{eq:2corr-def} 一样的定义,$4$阶归一化内积可以定义为:
\begin{equation}
    G_4^{(l)} = \frac{1}{n_l^2} \sum_{j,k=1}^{n_l} 
        \mathbb{E}[z_j^{(l)}z_j^{(l)}z_k^{(l)}z_k^{(l)}].
\end{equation}
从$l$层到$l+1$层的$4$阶correlator可以进行如下推导:
\begin{equation}
    \begin{aligned}
        & \mathbb{E}[z_{i_1}^{(l+1)}z_{i_2}^{(l+1)}z_{i_3}^{(l+1)}z_{i_4}^{(l+1)}] \\
        =& \sum_{j_i,j_2,j_3,j_4=1}^{n_l} 
            \mathbb{E}[W_{i_1j_1}^{(l+1)}W_{i_2j_2}^{(l+1)} W_{i_3j_3}^{(l+1)}W_{i_4j_4}^{(l+1)}] 
            \mathbb{E}[z_{j_1}^{(l)}z_{j_2}^{(l)} z_{j_3}^{(l)}z_{j_4}^{(l)}] \\
        =& \frac{C_W^2}{n_l^2} \sum_{j_i,j_2,j_3,j_4=1}^{n_l} 
            [(\delta_{i_1i_2}\delta_{j_1j_2}\delta_{i_3i_4}\delta_{j_3j_4}
            +\delta_{i_1i_3}\delta_{j_1j_3}\delta_{i_2i_4}\delta_{j_2j_4}
            +\delta_{i_1i_4}\delta_{j_1j_4}\delta_{i_2i_3}\delta_{j_2j_3})] \\
        & \times \mathbb{E}[z_{j_1}^{(l)}z_{j_2}^{(l)}z_{j_3}^{(l)}z_{j_4}^{(l)}] \\
        = & C_W^2 (\delta_{i_1i_2}\delta_{i_3i_4} + \delta_{i_1i_3}\delta_{i_2i_4} 
            + \delta_{i_1i_4}\delta_{i_2i_3}) \frac{1}{n_l^2} 
            \sum_{j,k=1}^{n_l}\mathbb{E}[z_j^{(l)}z_j^{(l)}z_k^{(l)}z_k^{(l)}] \\
        = & C_W^2 (\delta_{i_1i_2}\delta_{i_3i_4} + \delta_{i_1i_3}\delta_{i_2i_4} 
            + \delta_{i_1i_4}\delta_{i_2i_3}) G_4^{(l)},
    \end{aligned}
    \label{eq:4point-recur-1}
\end{equation}
由此可得,
\begin{equation}
    \begin{aligned}
    G_4^{(l+1)} = & \frac{1}{n_{l+1}^2} \sum_{i,j=1}^{n_{l+1}}
        \mathbb{E}[z_i^{(l+1)}z_i^{(l+1)}z_j^{(l+1)}z_j^{(l+1)}] \\
        = & \frac{1}{n_{l+1}^2} C_W^2 \sum_{i,j=1}^{n_{l+1}}
            (\delta_{ii}\delta_{jj} + \delta_{ij}\delta_{ij} + \delta_{ij}\delta_{ji}) G_4^{(l)} 
        = C_W^2 (1 + \frac{2}{n_{l+1}}) G_4^{(l)}.
    \end{aligned}
\end{equation}
根据公式\ref{eq:4point-recur-1}, 从$0$到$1$层的推导类似:
\begin{equation*}
    \begin{aligned}
     & \mathbb{E}[z_{i_1}^{(1)}z_{i_2}^{(1)}z_{i_3}^{(1)}z_{i_4}^{(1)}] \\
    =& C_W^2 (\delta_{i_1i_2}\delta_{i_3i_4} + \delta_{i_1i_3}\delta_{i_2i_4} 
            + \delta_{i_1i_4}\delta_{i_2i_3}) \frac{1}{n_0^2} 
            \sum_{j,k=1}^{n_0} \mathbb{E}[x_jx_jx_kx_k] \\
    =& C_W^2 (\delta_{i_1i_2}\delta_{i_3i_4} + \delta_{i_1i_3}\delta_{i_2i_4} 
            + \delta_{i_1i_4}\delta_{i_2i_3}) \frac{1}{n_0^2} 
            \sum_{j=1}^{n_0} \mathbb{E}[x_jx_j] \sum_{k=1}^{n_0} \mathbb{E}[x_k x_k] \\
    =&  C_W^2 (\delta_{i_1i_2}\delta_{i_3i_4} + \delta_{i_1i_3}\delta_{i_2i_4} 
            + \delta_{i_1i_4}\delta_{i_2i_3}) (G_2^{(0)})^2
    \end{aligned}
\end{equation*}
这里用到了同一条数据不同维度$x_j$和$x_k$互不相关的假定。同样地,可以得到第一层$4$-point correlator为:
\begin{equation}
    G_4^{(1)} = C_W^2 (1+ \frac{2}{n_1}) (G_2^{(0)})^2
    \label{eq:4point-0}
\end{equation}
综合公式\ref{eq:4point-0}和公式\ref{eq:4point-recur}, 可以得到:
\begin{equation}
    G_4^{(l)} = C_W^{2l}[\prod_{l'=1}^l(1+\frac{2}{n_{l'}})](G_2^{(0)})^2
        = \prod_{l'=1}^l(1+\frac{2}{n_{l'}})(G_2^{(l)})^2.
    \label{eq:4point-recur}
\end{equation}
第二个等号用到了公式\ref{eq:recursion}的结果。

\subsubsection{物理学解释}
根据公式\ref{eq:4point-recur}, 如果是一个无限宽的网络 ($n_l \to \infty$), 那么递推公式将
变成 $G_4^{(l)} = (G_2^{(l)})^2$,full four-point correlator 将变成:
\begin{equation*}
    \mathbb{E}[z_{i_1}^{(l)}z_{i_2}^{(l)}z_{i_3}^{(l)}z_{i_4}^{(l)}]
        = (\delta_{i_1i_2}\delta_{i_3i_4} + \delta_{i_1i_3}\delta_{i_2i_4}
            + \delta_{i_1i_4}\delta_{i_2i_3}) (G_2^{(l)})^2.
\end{equation*}
This is exactly what we'd find if the preactivation distribution were Gaussian: 
the four-point correlator is determined entirely by the two-point correlator, 
with the tensor structure determined by Wick's theorem. 这就表示同一层原子之间相互独立,
且可以用高斯分布进行描述。

实际神经网络肯定是有限宽度,那么就无法完全服从上述高斯分布了,为了方便推导,可以假定
$n_1 = n_2 = \cdots = n_{L-1} = n$. 然后公式\ref{eq:4point-recur}变成:
\begin{equation}
    \begin{aligned}
    G_4^{(l)} - (G_2^{(l)})^2 =& \left[\left( 1 + \frac{2}{n} \right)^{l-1}
        - 1\right]\left(G_2^{(l)}\right)^2 \\
        =& \frac{2(l-1)}{n}\left(G_2^{(l)}\right)^2 + O\left(\frac{1}{n^2}\right).
    \end{aligned}
    \label{eq:4point-deviation}
\end{equation}
这里用到了对$1/n$在$0$处的泰勒展开: 
$f(x)=(1+2x)^{l-1} = f(0) + f'(0)x + O(x^2) = 1 + (l-1)2x + O(x^2)$.
由此,第一章中定义的the connected four-point correlator (sec. \ref{sec:prob-corr-stat})
可以表示为:
\begin{equation}
    \mathbb{E}\left[ z_{i_1}^{(l)}z_{i_2}^{(l)}z_{i_3}^{(l)}z_{i_4}^{(l)} \right] |_{connected}
        = (\delta_{i_1i_2}\delta_{i_3i_4} + \delta_{i_1i_3}\delta_{i_2i_4}
            + \delta_{i_1i_4}\delta_{i_2i_3}) \left[G_4^{(l)} - (G_2^{(l)})^2\right].
\end{equation}

1). \textbf{从$4$-point connected correlator的角度来理解},结合公式\ref{eq:4point-deviation}
和section \ref{sec:2point-ana}中的分析,
在临界状态下 ($C_W=1$),$G_2^{(l)}$是一个常数, \emph{这个$4$阶connected correlator的值取决于
depth-to-width ratio $l/n$. 值越大,各层原子离独立高斯分布的gap越大,即随着网络加深,
原子分布的non-Gaussianity也逐渐增大。 如果$l/n$比较小的话,preactivation statistics in 
layer $l$ 还是可以维持在一个nearly-Gaussian的水平。}

2). \textbf{从$4$阶协方差的角度来理解},\emph{公式\ref{eq:4point-deviation} controls intralayer 
interactions between distinct neurons, with the strength of the interactions 
growing with depth.} 层数越深,$l/n$越大,各原子之间的耦合也越强。

3). \textbf{从观测变量抖动 (fluctutations)的角度} (some observables that are 
deterministic in the infinite-width limit start to fluctuate at finite width).
首先从一个最简单的观测变量$l$层输出的平均幅度来看:
\begin{equation}
    \mathcal{O}^{(l)} \equiv \mathcal{O}(z^{(l)}) \equiv 
        \frac{1}{n} \sum_{j=1}^n z_j^{(l)}z_j^{(l)},
\end{equation}
which captures the average magnitude of the preactivations over all the different
neurons in a hidden layer $l$ for a given instantiation of the network weights.
不同weights下的输出幅度均值可以计算如下:
\begin{equation*}
    \mathbb{E}\left[\mathcal{O}^{(l)}\right] = 
        \frac{1}{n}\sum_{j=1}^n \mathbb{E}[z_j^{(l)}z_j^{(l)}] = G_2^{(l)},
\end{equation*}
不同weights instantiation下该观察变量的抖动幅度可以通过其方差来度量:
\begin{equation}
    \begin{aligned}
        \mathbb{E}\left[\left(\mathcal{O}^{(l)} - \mathbb{E}\left[
            \mathcal{O}^{(l)} \right]\right)^2\right]
        =& \frac{1}{n^2} \sum_{j,k=1}^n 
            \mathbb{E}\left[z_j^{(l)}z_j^{(l)}z_k^{(l)}z_k^{(l)}\right]
            - \left(G_2^{(l)}\right)^2 \\
        =& G_4^{(l)} - \left(G_2^{(l)}\right)^2 
        = \frac{2(l-1)}{n}\left(G_2^{(l)}\right)^2 + O\left(\frac{1}{n^2}\right).
    \end{aligned}
\end{equation}
上式用到了公式 \ref{eq:4point-deviation}的结论。 这里可以看出,在临界状态的有限宽度的神经网络中,
方差随着深度的加深线性增加。这就表明,深度加深,意味着typical magnitude of the preactivations 
$\mathcal{O}(l)$ measured on any given realization of the deep linear network may 
deviate more from the mean value $\mathbb{E}\left[O^{(l)}\right] = G_2^{(l)}$, 即
输出的抖动 (finite-width fluctuations) 越大。

1), 2), 3)表明三个维度 - non-Gaussianities, intralayer interactions, or finite-width 
fluctuations都是和网络的depth-to-width ratio成正比。 \emph{Thus, the deeper a network is, 
the less the infinite-width Gaussian description will apply, due to accumulation of 
finite-width fluctuations. This is actually a good thing because, infinite-width 
networks do not have correlations among neurons within a layer and cannot learn 
nontrivial representations from input data. Real useful deep learning systems 
that are used in practice do both of these things, and our later analysis will 
show that deeper networks have the capacity to do more of these things.}

Quite generally, in the regime where perturbation theory works, the finite-width 
corrections grow linearly - not exponentially - with depth and the network 
remains well behaved. By contrast, when the depth-to-width ratio becomes large, 
perturbation theory breaks down, making it very difficult to analyze such networks.