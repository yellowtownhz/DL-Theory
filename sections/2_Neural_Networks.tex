\subsection{Function Approximation}

\subsubsection{MLP Introduction}
\begin{quote}
    \textit{MLPs will play the role of archetype network architecture for illustrating the principles of deep learning throughout the book. This class of neural-network models is rich enough to capture all the essential aspects of deep learning theory, while simple enough to maintain the pedagogical focus of the book. Nevertheless, we'll also briefly comment on how one could work out an effective theory for other network architectures.}

    \rightline{---Page $35$, Paragraph $2$}
\end{quote}
MLP 是最基本的神经网络,本书着重研究MLP中的深度学习理论,也会简短介绍如何将MLP中的理论推广到其他结构的神经网络。

\subsubsection{Function Approximation}
一个机器学习模型本质上是一个函数$y=f(x)$,比如说,一个猫狗分类模型可以看做是一个从输入图像$x$,到输出概率$y$的映射函数。不同的模型对应的函数$f(x;\theta)$具有不同的参数$\theta$,当我们调整参数来提升模型的性能的时候,本质上是在通过调整函数的参数来接近那个最优的函数:$f(x;\theta^*)\approx f(x)$,这个函数估计的过程就叫做一个``机器学习算法(Learning Algorithm)''

\subsubsection{MLP Architecture}
MLP的计算过程为:
\begin{equation}
    z_i^{(1)}(x_\alpha)\equiv b_i^{(1)}+\sum^{n_0}_{j=1}W_{ij}^{(1)}x_{j:\alpha}
\end{equation}
\begin{equation}
    z_i^{(l+1)}(x_\alpha)\equiv b_i^{(l+1)}+\sum^{n_l}_{j=1}W_{ij}^{(l+1)}\sigma(z_j^{(l)}(x_\alpha))
\end{equation}
其中,MLP一共有$L$层,每一层$l$有$n_l$个神经元,$\sigma(z)$为激活函数

\subsubsection{Activation Functions}
\begin{figure}[!ht]
    \includegraphics[width=\textwidth]{images/2_activation_function.bmp}
    \caption{经常用到的9种激活函数}
    \label{fig:2-activation}
\end{figure}
图\ref{fig:2-activation}中展示了经常用到的九种激活函数。

第一种perceptron是一种fire or not的开关形式,问题在于传递的信息只有二进制的两种状态,信息太少。

第二种sigmoid是平滑版本的perceptron,输入的绝对值越大,和perceptron越接近。

第三种tanh是放缩加平移版本的sigmoid。

第五种linear和第六种ReLU以及leaky ReLU有一个共同的性质:尺度等变性(原文这里叫scale-invariant,按照意思应该是等变性才对)。
\begin{equation}
    \sigma(\lambda z)=\lambda \sigma(z)
\end{equation}
满足这个性质的函数只有一种形式(可以通过对$z$求导$\sigma^{'}(\lambda z)=\sigma^{'}(z)$证明):
\begin{equation}
    \sigma(z)=\left\{
        \begin{aligned}
            a_{+}z, z\geq 0,\\
            a_{-}z, z< 0.
        \end{aligned}
    \right.
\end{equation}
其中包括linear($a_{+}=a{-}-a$),ReLU($a_{+}=1, a_{-}=0$)和leaky ReLU($a_{+}=1, a_{-}=a$)。

第七种softplus函数,在$z\gg 1$的时候约等于$z$,在$z<0$的时候以指数级趋近于零($\sigma(z)\approx e^{-|z|}$),$\sigma(0)=log(2)$
\begin{equation}
    \sigma(z)=log(1+e^z)
\end{equation}

第八种SWISH函数是sigmoid函数乘上了一个$z$,在$z>0$的时候就像$\sigma(z)\approx z$,在$z<0$的时候就想$\sigma(z)\approx 0$,并且,乘零的操作保证了$\sigma(0)=0$
\begin{equation}
    \sigma(z)=\frac{z}{1+e^{-z}}
\end{equation}

最后一种GeLU(Gaussian Error Linear Unit)函数长的和SWISH函数差不多:
\begin{equation}
    \sigma(z)=[\frac{1}{2}+\frac{1}{2}erf(\frac{z}{\sqrt{2}})]\times z
\end{equation}
性质和SWISH函数差不多,其中erf$(z)$函数长的和tanh函数差不多。

最后三种平滑版本ReLU的函数都不符合尺度等变性。

\subsubsection{Initialization Distribution of Biases and Weights}
为什么深度学习中在初始化MLP的bias和weight的时候都会选择高斯分布?

定义一个分布$p(x)$的信息熵为:
\begin{equation}
    H[p]=-\int p(x)\log p(x)dx
\end{equation}
在满足$p(x)$积分为1,方差固定,均值固定三个约束条件下,可以转换成拉格朗日函数形式:
\begin{equation}
    \begin{aligned}
    \mathcal{L}[p]=&\lambda_1(\int p(x)dx-1)+\lambda_2(\mathbb{E}[x]-\mu)+\lambda_3(\mathbb{E}[(x-\mu)^2]-\sigma^2)+H[p]\\
    =&\int(\lambda_1p(x)+\lambda_2p(x)x+\lambda_3p(x)(x-\mu)^2-p(x)\log p(x))dx-\lambda_1-\\
    &\mu\lambda_2-\sigma^2\lambda_3
    \end{aligned}
\end{equation}
对$p$求导,并使其为0,可得:
\begin{equation}
    \forall x,\frac{\delta}{\delta p(x)}\mathcal{L}=\lambda_1+\lambda_2x+\lambda_3(x-\mu)^2-1-\log p(x)=0
\end{equation}
获得$p(x)$的最优解:
\begin{equation}
    p(x)=\exp (-\lambda_1-\lambda_2x+\lambda_3(x-\mu)^2+1)
\end{equation}
在满足三个限定条件的情况下,设定参数$\lambda_1=log \sigma\sqrt{2x}, \lambda_2=0, \lambda_3=-\frac{1}{2\sigma}$可获得$p(x)$的形式:$p(x)=\mathcal{N}(x;\mu,\sigma^2)$

