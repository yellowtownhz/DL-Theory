在第三章中,分析了Deep Linear Network(Sec. \ref{sec:3-1})中层与层之间前传的统计规律。
本章的目的是利用第三章讲到的的临界状态(criticality, Sec. \ref{sec:3})、扰动(fluctuation, Sec. \ref{sec:3})和第四章的Effective Theory(Sec. \ref{sec:4}),把MLP中结论推广到任意激活函数中,寻找合适的初始化条件。

\subsection{Criticality Analysis of the Kernel}
在第三章得到一个结论,在MLP中,两个输入$(x_{\alpha_1}, x_{\alpha_2})$协方差$G_{\alpha_1\alpha_2}^{(0)}$传递到$l$层的幅度为:
\begin{equation}
    G_{\alpha_1\alpha_2}^{(l)} = (C_W)^l G_{\alpha_1\alpha_2}^{(0)}.
\end{equation} 
如果我们用kernel来表示G,并且考虑bias,第$l$层和第$l+1$层之间的传递公式为:
\begin{equation}
    K_{\alpha\beta}^{(l+1)} = C_b + C_W\langle\sigma_\alpha \sigma_\beta\rangle_{K^{(l)}}
    \label{eq:kernel_recursion}
\end{equation} 
在第一层的初始状态为:
\begin{equation}
    K_{\alpha\beta}^{(1)} = C_b + C_W(\frac{1}{n_0}\sum^{n_0}_{i=1}x_{i;\alpha}x_{j;\beta})
\end{equation}
$K_{\alpha\beta}$的统计含义就是两个输入$\alpha$和$\beta$之间的协方差
我们的目的就是分析$K_{\alpha\beta}$如何随着层数的加深而不断变化的。

\subsubsection{Single Input}
首先我们考虑只有一个输入样本的情况,定义
\begin{equation}
    \begin{aligned}
    K_{00}^{(l+1)} &= C_b + C_W\langle\sigma(z_0)\sigma(z_0)\rangle_{K^{(l)}} \\
    &= C_b + C_Wg(K_{00}^{(l)})
    \end{aligned}
\label{eq:4-1-1}
\end{equation}
其中$g$为定义的一个辅助函数:
\begin{equation}
    g(K)\equiv \langle\sigma(z)\sigma(z)\rangle_{K} \equiv \frac{1}{\sqrt{2\pi K}}\int^{\inf}_{\inf}dz e^{-\frac{z^2}{2K}}\sigma(z)\sigma(z),
\end{equation}
第$l$层kernel的定义为:
\begin{equation}
    K_{00}^{(l)} = \mathbb{E}[\frac{1}{n_{l}}\sum^{n_{l}}_{i=1}(z^{(l)}_{i;0})^2]
\end{equation}

我们研究的是一个``固定点''$K_{00}^{*}$,使得Eq. \ref{eq:4-1-1}在这个固定点处,输出等于输入,即:
\begin{equation}
    K_{00}^{*} = C_b + C_Wg(K_{00}^{*})
\end{equation}
在$K_{00}^{*}$处微分化:
\begin{equation}
    K_{00}^{(l)} = K_{00}^{*} + \delta K_{00}^{l}
\end{equation}
\begin{equation}
    \delta K_{00}^{(l+1)} = \chi_\parallel(K_{00}^{*})\delta K_{00}^{*} + O(\delta^2)
\label{eq:4-1-2}
\end{equation}
其中,$\chi_\parallel(K_{00}^{*})$被称作parallel susceptibility。当$\chi_\parallel(K_{00}^{*})<1$时,特征的协方差会随着逐层前传而逐渐减小,直至消失;当$\chi_\parallel(K_{00}^{*})>1$时,特征协方差会逐渐变大,直至爆炸。所以只有\textbf{临界条件}$\chi_\parallel(K_{00}^{*})=1$满足的时候,特征的协方差才会稳定。

\subsubsection{Two Inputs}
当有两个样本的时候,Eq. \ref{eq:4-1-2}变得更复杂:
\begin{equation}
    \delta \delta K_{[2]}^{(l+1)} = \chi_\perp(K_{00}^{(l)})\delta \delta K_{[2]}^{l} + h(K_{00}^{(l)})(\delta K_{[1]}^{(l)})^2
\label{eq:4-1-3}
\end{equation}
此时,临界条件变为:
\begin{equation}
    \chi_\parallel(K_{00}^{*})=1, \qquad \chi_\perp(K_{00}^{*})=1
\label{eq:4-1-4}
\end{equation}

\subsection{Criticality for Scale-Invariant Activations}
回忆第二章中对scale-invariant函数的定义:
\begin{equation}
    \sigma(z)=\left\{
        \begin{aligned}
            a_{+}z, z\geq 0,\\
            a_{-}z, z< 0.
        \end{aligned}
    \right.
\end{equation}
在scale-invariant函数的条件下,辅助函数$g$变得非常简单:
\begin{equation}
    g(K)=A_2 K, \qquad A_2\equiv \frac{a_{+}^2+a_{-}^2}{2},
\end{equation}
$A_2$是由scale-invariant函数形式决定的常数。数学推导可得,这种条件下两个susceptibility相等,并且和$K_{00}^{(l)}$独立:
\begin{equation}
    \chi_\parallel(K_{00}^{(l)})=\chi_\perp(K_{00}^{(l)})=A_2 C_W \equiv \chi
\end{equation}
当$\chi>1$时,特征幅度会逐渐爆炸;当$\chi<1$时,特征幅度会逐渐消失;只有当$C_W=1/A_2, C_b=0$时,状态为临界状态,此时:
\begin{equation}
    K_{00}^{*}=\frac{1}{A_2}(\frac{1}{n_0}\sum^{n_0}_{i=1}x_{i;0}^2)
\end{equation}

总而言之,deep linear network的临界条件为:
\begin{equation}
    (C_b, C_W)^{critical}=(0, \frac{1}{A_2}), \qquad A_2=(a_{+}^2+a_{-}^2)/2
\end{equation}
当激活函数为$\mathtt{ReLU}$时,这个形式就等同于Kaiming初始化$(C_b, C_W)^{critical}=(0, 2)$

\subsection{Universality beyond Scale-Invariant Activations}
本小节简单推导其他激活函数有没有临界状态,有的话临界状态需要满足哪些条件。

从上一小节我们可以得出判断临界状态的两步走:
\begin{itemize}
    \item 第一步,对于每一对$C_b$和$C_W$,找到临界点$K_{00}^*=K_{00}^*(C_b, C_W)$,使得$K_{00}^*=C_b+C_W g_0(K_{00}^*)$,同时$K_{00}^*\ge0$
    \item 第二步,对于找到的每一个临界点$K_{00}^*=K_{00}^*(C_b, C_W)$,计算$\chi_\parallel(K_{00}^*)$和$\chi_\perp(K_{00}^*)$,寻找满足临界条件$\chi_\parallel(K_{00}^{*})=1, \qquad \chi_\perp(K_{00}^{*})=1$的$(C_b, C_W)$
\end{itemize}

\subsubsection{$\texttt{tanh}$}
\begin{figure}[!ht]
    \includegraphics[width=\textwidth]{images/4_tanh.bmp}
    \caption{
        两种计算$\mathtt{tanh}$函数临界条件的算法。\textbf{左图}:横坐标表示$C_W$,纵坐标表示$C_b$,实线表示的是满足$\chi_\perp(K_{00}^{*})=1$的$(C_b, C_W)$值,虚线表示的是满足$\chi_\parallel(K_{00}^{*})=1$的$(C_b, C_W)$值。两条线相交在$(C_b, C_W)=(1,0)$处。\textbf{右图}:$LHS$曲线,可以看到满足$LHS=1$时,$K_{00}^*\to0$
    }
    \label{fig:4-tanh}
\end{figure}
有图\ref{fig:4-tanh}左图可以看到,$\texttt{tanh}$的临界状态为$(C_b, C_W)=(1,0)$。

在实际使用中,上述算法计算过于繁琐,所以改为第二种更容易操作算法:
\begin{itemize}
    \item 第一步,搜索$K_{00}^{*}\ge1$,找出满足$LHS\equiv\left[\frac{2K^2 \langle\sigma^{'}(z)\sigma^{'}(z)\rangle_K}{\langle\sigma(z)\sigma(z)(z^2-K)\rangle_K}\right]\arrowvert_{K=K_{00}^*}=1$的$K_{00}^{*}$
    \item 第二步,把找到的$K_{00}^{*}$代入公式$C_W=[\langle\sigma^{'}(z)\sigma^{'}(z)\rangle_{K_{00}^*}]^{-1}$和$C_b=K_{00}^*-\frac{\langle\sigma(z)\sigma(z)\rangle_{K_{00}^*}}{\langle\sigma^{'}(z)\sigma^{'}(z)\rangle_{K_{00}^*}}$,得到满足临界条件的初始化参数(保证$C_b\ge0, C_W\ge 0$)。
\end{itemize}
在图\ref{fig:4-tanh}右图中可以看到,满足临界条件时,$K_{00}^*\to0$,代入公式可得:
\begin{equation}
    C_W=\left[\langle\sigma^{'}(z)\sigma^{'}(z)\rangle_{K_{00}^*}\right]^{-1}=(\sigma^{'}(0))^{-2}=1,
\end{equation}
\begin{equation} 
    C_b=K_{00}^*-\frac{\langle\sigma(z)\sigma(z)\rangle_{K_{00}^*}}{\langle\sigma^{'}(z)\sigma^{'}(z)\rangle_{K_{00}^*}}=-\left(\frac{\sigma(0)}{\sigma^{'}(0)}\right)^2=0
\end{equation}

\subsubsection{$\texttt{sigmoid}$}
\begin{figure}[!ht]
    \includegraphics[width=\textwidth]{images/4_sigmoid_softplus.bmp}
    \caption{
        $\texttt{sigmoid}$和$\texttt{softplus}$的$LHS$曲线。
    }
    \label{fig:4-sigmoid}
\end{figure}
$\texttt{sigmoid}$的公式为:
\begin{equation}
    \sigma(z)=\frac{1}{1+\exp^{-z}}
\end{equation}
由图\ref{fig:4-sigmoid}可以看到,$\texttt{sigmoid}$在$K_{00}^*=0$时,$LHS=1$,满足临界条件。但是此时$C_b=-\left(\frac{\sigma(0)}{\sigma^{'}(0)}\right)^2$,所以$\texttt{sigmoid}$函数没有临界状态。

\subsubsection{$\texttt{softplus}$}
\begin{figure}[!ht]
    \centering
    \begin{subfigure}
    {
        \includegraphics[width=0.4\textwidth]{images/4_softplus_func.bmp}
        \label{fig:4-softplus-func}
    }
    \end{subfigure}
    \begin{subfigure}
    {
        \includegraphics[width=0.45\textwidth]{images/4_softplus_LHS.bmp}
        \label{fig:4-softplus-LHS}
    }
    \end{subfigure}
    \caption{$\texttt{softplus}$的函数曲线$LHS$曲线}
\end{figure}

$\texttt{softplus}$的计算公式为:
\begin{equation}
    \sigma(z)=log(1+e^z),
\end{equation}
函数曲线为图\ref{fig:4-softplus-func}。从图\ref{fig:4-softplus-LHS}可以看到,$\texttt{softplus}$的$LHS$曲线没有等于1的情况,所以尽管$\texttt{softplus}$函数从样子上属于$\texttt{ReLU}$的平滑版本,但是依旧没有临界状态。

\subsubsection{$\texttt{SWISH}$ and $\texttt{GELU}$}
分析略去,结论:
如果要使用平滑版本的$\texttt{ReLU}$函数,$\texttt{SWISH}$和$\texttt{GELU}$都不是一个好的选择,推荐使用$\texttt{tanh}$。

总的来说,给定一种非线性函数,利用这套临界函数的理论可以计算出以这种函数作为激活函数的神经网络的初始化条件。

\subsection{Fluctuations}
在非线性激活函数条件下,主要结论和Sec. \ref{sec:3-3}一致:
\begin{itemize}
    \item 在deep linear networks中,有限宽网络带来的方差抖动(finite-width fluctuations)会随着depth-to-width的比值增大而增大。这个结论在和非线性激活函数的网络同样适用。
    \item 在$L/n$固定,$n,L\to \infty$的情况下,绝大部分足够宽、不是太深的网络都可以不用考虑NLO metric $G_{\alpha_1\alpha2}^{\{1\}\{l\}}$的作用
\end{itemize}