\subsection{First Layer: Good-Old Gaussian}
先给出一些基本的notations. 数据集 ($N_D$个样本,每个样本$n_0$维):
\begin{equation}
    \mathcal{D} = \{x_{i;\alpha}\}_{i=1,...,n_0;\alpha=1,...,N_D}.
\end{equation}
第一层preactivations:
\begin{equation}
    z_{i;\alpha}^{(1)} \equiv z_i^{(1)}(x_\alpha) = b_i^{(1)} + 
        \sum_{j=1}^{n_0}W_{ij}^{(1)}x_{j;\alpha},~ for ~ i = 1,...,n_1.
    \label{eq:ch4-preact}
\end{equation}
$W^{(1)}$和$b^{(1)}$均为零均值高斯分布,方差如下:
\begin{equation}
    \mathbb{E}\left[b_i^{(1)}b_j^{(1)}\right] = \delta_{ij}C_b^{(1)},
    \label{eq:ch4_var_b}
\end{equation}
\begin{equation}
    \mathbb{E}\left[W_{i_1j_1}^{(1)}W_{i_2j_2}^{(1)}\right] = 
        \delta_{i_1i_2}\delta_{j_1j_2}\frac{C_W^{(1)}}{n_0}.
    \label{eq:ch4_var_w}
\end{equation}
关注的分布可以表示为:
\begin{equation}
    p(z^{(1)}|\mathcal{D}) = p(z^{(1)}(x_1),...,z^{(1)}(x_{N_D})).
\end{equation}

\subsubsection{Combinatorial derivation via correlators}
对于one-point correlator,根据preactivations定义 Eq. \ref{eq:ch4-preact},
\begin{equation}
    \mathbb{E}\left[z_{i;\alpha}^{(1)}\right] = 
        \mathbb{E}\left[b_i^{(1)} + \sum_{j=1}^{n_0}W_{ij}^{(1)}x_{j;\alpha_1}\right]
        = 0,
\end{equation}
这里用到了$\mathbb{E}\left[b_i^{(1)}\right] =0$ 和 $\mathbb{E}\left[W_{ij}^{(1)}\right] =0$.
同理,所有奇数个point的correlators,由于在进行Wick contraction组成两两pair对以后,总会剩余一个单独的$b_i$
或者单独的$W_{ij}$,所以都会为$0$.

接下来计算two-point correlator,
\begin{equation}
    \begin{aligned}
        \mathbb{E}\left[z_{i_1;\alpha_1}^{(1)}z_{i_2;\alpha_2}^{(1)}\right]
        =& \mathbb{E}\left[\left(b_{i_1}^{(1)} + 
        \sum_{j_1=1}^{n_0}W_{i_1j_1}^{(1)}x_{j_1;\alpha_1} \right)
        \left(b_{i_2}^{(1)} + \sum_{j_2=1}^{n_0}W_{i_2j_2}^{(1)}x_{j_2;\alpha_2} \right)\right] \\
        =& \delta_{i_1i_2}\left(C_b^{(1)} + C_W^{(1)}\frac{1}{n_0}
            \sum_{j=1}^{n_0} x_{j;\alpha_1}x_{j;\alpha_2}\right) 
        = \delta_{i_1i_2} G_{\alpha_1\alpha_2}^{(1)},
    \end{aligned}
    \label{eq:ch4_2point_corr}
\end{equation}
这里先把第一行的括号展开,会有$4$项,其中两项为括号里面第一项和第一项组合,第二项和
第二项组合,也就是
\begin{equation*}
    \mathbb{E} \left(b_{i_1}^{(1)} b_{i_2}^{(1)} \right) + 
    \mathbb{E} \left[\left(\sum_{j_1=1}^{n_0}W_{i_1j_1}^{(1)}x_{j_1;\alpha_1}\right)
    \left(\sum_{j_2=1}^{n_0}W_{i_2j_2}^{(1)}x_{j_2;\alpha_2} \right)\right]
\end{equation*}
为非$0$,其他两项都会存在$b$和$W$为奇数个的情况,均为$0$. 第二项只有$j_1 = j_2$才为非$0$,进一步化简为
\begin{equation*}
    \mathbb{E} \left(b_{i_1}^{(1)} b_{i_2}^{(1)} \right) + 
    \mathbb{E} \left( \sum_{j=1}^{n_0}W_{i_1j}^{(1)}
        W_{i_2j}^{(1)}x_{j;\alpha_1}x_{j;\alpha_2} \right)
\end{equation*}
代入Eq. \ref{eq:ch4_var_b}和Eq. \ref{eq:ch4_var_w}即可得到Eq. \ref{eq:ch4_2point_corr}中最后一行的结果。
这里引入了first-layer metric (\xu{TODO: metric 的物理含义是啥?@树贤})
\begin{equation}
    G_{\alpha_1\alpha_2}^{(1)} = C_b^{(1)} + C_W^{(1)}\frac{1}{n_0}
            \sum_{j=1}^{n_0} x_{j;\alpha_1}x_{j;\alpha_2}.
    \label{eq:ch4_layer1_metric}
\end{equation}
该定义为两个不同样本之间的two-point correlation of preactivations.
同理,four-point correlator 可以类似地代入Eq. \ref{eq:ch4-preact}对$W$和$b$
使用Wick-contracting,得到:
\begin{equation}
    \begin{aligned}
       & \mathbb{E}\left[z_{i_1;\alpha_1}^{(1)}z_{i_2;\alpha_2}^{(1)}
        z_{i_3;\alpha_3}^{(1)}z_{i_4;\alpha_4}^{(1)}\right] \\
       =& \delta_{i_1i_2}\delta_{i_3i_4}G_{\alpha_1\alpha_2}^{(1)}G_{\alpha_3\alpha_4}^{(1)}
       + \delta_{i_1i_3}\delta_{i_2i_4}G_{\alpha_1\alpha_3}^{(1)}G_{\alpha_2\alpha_4}^{(1)}
       + \delta_{i_1i_4}\delta_{i_2i_3}G_{\alpha_1\alpha_4}^{(1)}G_{\alpha_2\alpha_3}^{(1)} \\
       =& \mathbb{E}\left[z_{i_1;\alpha_1}^{(1)}z_{i_2;\alpha_2}^{(1)}\right]
          \mathbb{E}\left[z_{i_3;\alpha_3}^{(1)}z_{i_4;\alpha_4}^{(1)}\right]
       + \mathbb{E}\left[z_{i_1;\alpha_1}^{(1)}z_{i_3;\alpha_3}^{(1)}\right]
          \mathbb{E}\left[z_{i_2;\alpha_2}^{(1)}z_{i_4;\alpha_4}^{(1)}\right] \\
       & + \mathbb{E}\left[z_{i_1;\alpha_1}^{(1)}z_{i_4;\alpha_4}^{(1)}\right]
          \mathbb{E}\left[z_{i_2;\alpha_2}^{(1)}z_{i_3;\alpha_3}^{(1)}\right]
    \end{aligned}
\end{equation}
最后一行表达式等价于对$4$元组方差进行Wick-contracting展开. 根据第一章中对connected four-point correlator
的定义可得:
\begin{equation}
    \mathbb{E}\left[z_{i_1;\alpha_1}^{(1)}z_{i_2;\alpha_2}^{(1)}
    z_{i_3;\alpha_3}^{(1)}z_{i_4;\alpha_4}^{(1)}\right] \Big|_{connected} = 0.
\end{equation}
同理,其他高阶的connected correlators也可以通过类似Eq. \ref{eq:ch4-preact}展开得到,并且都为$0$. 
这也就是说所有的correlators都来自于一个零均值方差为Eq. \ref{eq:ch4-preact}的高斯分布.

为了得到第一层的action,需要求解该variance的逆,由满足下式的$\delta_{i_1i_2}G^{\alpha_1\alpha_2}_{(1)}$给出:
\begin{equation}
    \sum_{j=1}^{n_1} \sum_{\beta\in D} \left(\delta_{i_1j}G_{(1)}^{\alpha_1\beta}\right)
    \left(\delta_{ji_2}G^{(1)}_{\beta\alpha_2}\right) = \delta_{i_1i_2}\delta_{\alpha_1}^{\alpha_2},
\end{equation}
由第一章中的定义,$G_{(1)}^{\alpha_1\beta}$为$G^{(1)}_{\alpha_1\beta}$的逆,满足:
\begin{equation}
    \sum_{\beta\in D}G_{(1)}^{\alpha_1\beta}G^{(1)}_{\alpha_1\beta}=\delta^{\alpha_1}_{\alpha_2}.
\end{equation}
有了上述的逆, 第一层preactivations的高斯分布可以被表达为:
\begin{equation}
    p\left(z^{(1)}\bigg| D\right) = \frac{1}{Z}e^{-S(z^{(1)})},
    \label{eq:ch4_z1_p}
\end{equation}
这里二次的action $S$为:
\begin{equation}
    S(z^{(1)}) = \frac{1}{2}\sum_{i=1}^{n_1}\sum_{\alpha_1,\alpha_2\in D} 
        G_{(1)}^{\alpha_1\beta}z_{i;\alpha_1}^{(1)}z_{i;\alpha_2}^{(1)},
\end{equation}
partition function $Z$为:
\begin{equation}
    Z = \int \left[\prod_{i,\alpha} dz_{i;\alpha}^{(1)}\right]e^{-S(z^{(1)})}
        = \big|2\pi G^{(1)}\big|^{\frac{n_1}{2}}.
\end{equation}
这里$\big|2\pi G^{(1)}\big|$是$N_D$-by-$N_D$矩阵$2\pi G_{\alpha_1\alpha_2}^{(1)}$
的行列式。 \emph{$S$, $Z$ 等的计算均在第一章中有详细推导和定义。}

\subsection{Second Layer: Genesis of Non-Gaussianity}
考虑MLP的第二层preactivations的分布情况,
\begin{equation}
    z_{i;\alpha}^{(2)} \equiv z_{i}^{(2)}(x_\alpha) = b_i^{(2)} + \sum_{j=1}^{n_1}
        W_{ij}^{(2)}\sigma_{j;\alpha}^{(1)},~~ for ~~ i=1,...,n_2,
\end{equation}
其中第一层的激活值为:
\begin{equation}
    \sigma_{i;\alpha}^{(1)} \equiv \sigma(z_{i;\alpha}^{(1)}),
\end{equation}
并且$b^{(2)}$和$W^{(2)}$ 均服从高斯分布的采样。

一二两层preactivations的联合分布可以被分解为:
\begin{equation}
    p\left(z^{(2)},z^{(1)} \bigg| \mathcal{D} \right) = p\left(z^{(2)} \bigg| z^{(1)}\right)
        p\left(z^{(1)} \bigg| \mathcal{D} \right).
\end{equation}
$p\left(z^{(1)} \bigg| \mathcal{D} \right)$在上一节中已经解决,是一个标准的高斯分布,方差由first-layer
metric $G_{\alpha_1\alpha_2}^{(1)}$决定。第一项条件分布可以被拆解为:
\begin{equation}
    \begin{aligned}
    & p\left(z^{(2)} \bigg| z^{(1)}\right) \\
    =& \int \left[\prod_i db_i^{(2)}p(b_i^{(2)})\right]\left[\prod_{i,j}dW_{ij}^{(2)}
        p\left(W_{ij}^{(2)}\right)\right]\prod_{i,\alpha}\delta\left(z_{i;\alpha}^{(2)}
        -b_i^{(2)} - \sum_j W_{ij}^{(2)}\sigma_{j;\alpha}^{(1)}\right),
    \end{aligned}
\end{equation}
这里的$\delta$为指示函数,$\delta_{\mu v} = 1$ if $\mu = v$, else $0$. \xu{NOTE: 这个公式
需要补充一些解释}.

$z^{(2)}$的边缘分布可以直接通过$z^{(2)},z^{(1)}$联合分布对$z^{(1)}$进行积分得到:
\begin{equation}
    p\left(z^{(2)} \bigg| \mathcal{D} \right) = \int \left[\prod_{i,\alpha}
        dz_{i;\alpha}^{(1)}\right]p\left(z^{(2)}\big| z^{(1)}\right)
        p\left(z^{(1)}\big| \mathcal{D}\right).
\end{equation}
接下来先解决条件分布,然后解决积分的问题。

\subsubsection{Second-layer conditional distribution}
条件概率的计算方法与4.1中完全相同,只需要把Eq. \ref{eq:ch4_z1_p}中相关度量从$x$换成$\sigma^{(1)}$:
\begin{equation}
    p\left(z^{(2)}\big| z^{(1)}\right) = \frac{1}{\sqrt{\big|2\pi \hat{G}^{(2)}\big|^{n_2}}}
    exp\left(-\frac{1}{2}\sum_{i=1}^{n_2}\sum_{\alpha_1\alpha_2\in\mathcal{D}}
    \hat{G}^{\alpha_1,\alpha_2}_{(2)}z_{i;\alpha_1}^{(2)}z_{i;\alpha_2}^{(2)}\right),
    \label{eq:ch4_z2_cond_p}
\end{equation}
这里引入\emph{stochastic} second-layer metric
\begin{equation}
    \hat{G}^{\alpha_1,\alpha_2}_{(2)} \equiv C_b^{(2)} + C_W^{(2)}\frac{1}{n_1}
    \sum_{j=1}^{n_1}\sigma_{j;\alpha_1}^{(1)}\sigma_{j;\alpha_2}^{(1)},
    \label{eq:ch4_2layer_metric}
\end{equation}
相比于4.1中,$G$有hat是因为$G$有随机性,通过$\sigma^{(1)}\equiv \sigma(z^{(1)})$依赖于
stochastic 变量$z^{(1)}$。\emph{因此第二层条件分布也是一个variance为随机变量的高斯分布}。即the stochastic
second-layer metric fluctuates around the \emph{mean} second-layer metric
\begin{equation}
    \begin{aligned}
    G_{\alpha_1\alpha_2}^{(2)} \equiv \mathbb{E}\left[\hat{G}^{\alpha_1,\alpha_2}_{(2)}\right]
    = & C_b^{(2)} + C_W^{(2)}\frac{1}{n_1}\sum_{j=1}^{n_1}\mathbb{E}\left[\sigma_{j;\alpha_1}^{(1)}
    \sigma_{j;\alpha_2}^{(1)}\right] \\
    = & C_b^{(2)} + C_W^{(2)}\langle\sigma_{\alpha_1}\sigma_{\alpha_2}\rangle_{G^{(1)}},
    \end{aligned}
    \label{eq:ch4_2fluc}
\end{equation}
\xu{TODO: 第二行的推导公式涉及到高斯积分的计算比较难敲,暂时先这么用着,回头再补。}

基于此,定义fluctuation of the second-layer metric 为:
\begin{equation}
    \hat{\Delta G}_{\alpha_1\alpha_2}^{(2)} \equiv \hat{G}^{\alpha_1,\alpha_2}_{(2)} -
    G_{\alpha_1\alpha_2}^{(2)} = C_W^{(2)}\frac{1}{n_1} \sum_{j=1}^{n_1}
    \left(\sigma_{j;\alpha_1}^{(1)}\sigma_{j;\alpha_1}^{(1)} - 
    \langle\sigma_{\alpha_1}\sigma_{\alpha_2}\rangle_{G^{(1)}}\right),
    \label{eq:ch4_fluc_metric_layer2}
\end{equation}
显然该fluctuation期望为$0$.
\begin{equation}
    \mathbb{E}\left[\hat{\Delta G}_{\alpha_1\alpha_2}^{(2)}\right] = 0.
\end{equation}
上式为同一个原子的两个不同activations的variance,接下来研究两个不同原子的4阶activations的情况
\begin{equation}
    \begin{aligned}
    & \mathbb{E}\left[\hat{\Delta G}_{\alpha_1\alpha_2}^{(2)} 
    \hat{\Delta G}_{\alpha_3\alpha_4}^{(2)}\right] \\
    =& \left(\frac{C_W^{(2)}}{n_1}\right)^2\sum_{j,k=1}^{n_1}\mathbb{E}\left[\left(
    \sigma_{j;\alpha_1}^{(1)}\sigma_{j;\alpha_2}^{(1)} - \mathbb{E}\left[
    \sigma_{j;\alpha_1}^{(1)}\sigma_{j;\alpha_2}^{(1)} \right]\right)
    \left(\sigma_{k;\alpha_1}^{(1)}\sigma_{k;\alpha_2}^{(1)} - \mathbb{E}\left[
    \sigma_{k;\alpha_1}^{(1)}\sigma_{k;\alpha_2}^{(1)} \right]\right)\right] \\
    =& \left(\frac{C_W^{(2)}}{n_1}\right)^2\sum_{j}^{n_1}\left\{\mathbb{E}\left[
    \sigma_{j;\alpha_1}^{(1)}\sigma_{j;\alpha_2}^{(1)}
    \sigma_{j;\alpha_3}^{(1)}\sigma_{j;\alpha_4}^{(1)}\right] - 
    \mathbb{E}\left[ \sigma_{j;\alpha_1}^{(1)}\sigma_{j;\alpha_2}^{(1)}\right]
    \mathbb{E}\left[ \sigma_{j;\alpha_3}^{(1)}\sigma_{j;\alpha_4}^{(1)}\right] \right\} \\
    =& \frac{1}{n_1}\left(C_W^{(2)}\right)^2\left[\langle\sigma_{\alpha_1}\sigma_{\alpha_2} 
    \sigma_{\alpha_3}\sigma_{\alpha_4}\rangle_{G^{(1)}} - \langle\sigma_{\alpha_1}
    \sigma_{\alpha_2}\rangle_{G^{(1)}}
    \langle\sigma_{\alpha_1}\sigma_{\alpha_2}\rangle_{G^{(1)}}\right] \\
    \equiv & \frac{1}{n_1} V_{(\alpha_1\alpha_2)(\alpha_3\alpha_4)}^{(2)},
    \end{aligned}
    \label{eq:ch4_4point_vertex}
\end{equation}
这里用到了Eq. \ref{eq:ch4_2fluc}中的形式化定义,同时直接把第二行的括号展开合并即可得到第三行的结果。引入了
four-point vertex 的定义$V_{(\alpha_1\alpha_2)(\alpha_3\alpha_4)}^{(2)}=V\left(x_{\alpha_1},x_{\alpha_2};
x_{\alpha_3},x_{\alpha_4}\right)$,依赖于$4$个输入数据并且是关于下标对称 $\alpha_1 \leftrightarrow \alpha_2$, 
$\alpha_3 \leftrightarrow \alpha_4$ 并且 $(\alpha_1, \alpha_2) \leftrightarrow (\alpha_3, \alpha_4)$.
\xu{TODO: Vertex是啥意思?有啥物理含义?@树贤}. 

可以看出该metric fluctuation是关于$\frac{1}{n_1}$的一阶项,如果在网络足够宽的情况下 ($n_1 \gg 1$),条件分布
$(z^{(2)}\big| z^{(1)})$依然是一个确定的标准高斯分布。

\subsubsection{Wick: derivation for distributions of preactivations}
条件概率分布在上一节中已经解决,接下来先搞定$2$阶correlator和$4$阶correlator,然后就可以得到第$2$层preactivations
的分布情况了。

应用上一节的推导+Wick定理,可以得到second-layer preactivations的2-point correlator:
\begin{equation}
    \mathbb{E}\left[z_{i_1;\alpha_1}^{(2)}z_{i_2;\alpha_2}^{(2)}\right]
    = \delta_{i_1i_2}\mathbb{E}\left[\hat{G}_{\alpha_1\alpha_2}^{(2)}\right]
    = \delta_{i_1i_2}G_{\alpha_1\alpha_2}^{(2)}
    = \delta_{i_1i_2}\left(C_b^{(2)} + C_W^{(2)}\langle\delta_{\alpha_1}\delta_{\alpha_2}
    \rangle_{G^{(1)}}\right),
\end{equation}
这里用到了类似Eq. \ref{eq:ch4_2point_corr}的推导过程以及Eq. \ref{eq:ch4_2layer_metric}中的定义。

同样地,full four-point correlator也可以如下计算:
\begin{equation}
    \begin{aligned}
    & \mathbb{E}\left[z_{i_1;\alpha_1}^{(2)}z_{i_2;\alpha_2}^{(2)}
    z_{i_3;\alpha_3}^{(2)}z_{i_4;\alpha_4}^{(2)}\right] \\
    =& \delta_{i_1i_2}\delta_{i_3i_4}\mathbb{E}\left[\hat{G}_{\alpha_1\alpha_2}^{(2)}
    \hat{G}_{\alpha_3\alpha_4}^{(2)}\right] + 
    \delta_{i_1i_3}\delta_{i_2i_4}\mathbb{E}\left[\hat{G}_{\alpha_1\alpha_3}^{(2)}
    \hat{G}_{\alpha_2\alpha_4}^{(2)}\right] +
    \delta_{i_1i_4}\delta_{i_2i_3}\mathbb{E}\left[\hat{G}_{\alpha_1\alpha_4}^{(2)}
    \hat{G}_{\alpha_2\alpha_3}^{(2)}\right]\\
    =& \delta_{i_1i_2}\delta_{i_3i_4}G_{\alpha_1\alpha_2}^{(2)}G_{\alpha_3\alpha_4}^{(2)}+
    \delta_{i_1i_3}\delta_{i_2i_4}G_{\alpha_1\alpha_3}^{(2)}G_{\alpha_2\alpha_4}^{(2)}+
    \delta_{i_1i_4}\delta_{i_2i_3}G_{\alpha_1\alpha_4}^{(2)}G_{\alpha_2\alpha_3}^{(2)} \\
    &+ \frac{1}{n_1}\left[\delta_{i_1i_2}\delta_{i_3i_4}V_{(\alpha_1\alpha_2)(\alpha_3\alpha_4)}^{(2)}
     +\delta_{i_1i_3}\delta_{i_2i_4}V_{(\alpha_1\alpha_3)(\alpha_2\alpha_4)}^{(2)}
     +\delta_{i_1i_4}\delta_{i_2i_3}V_{(\alpha_1\alpha_4)(\alpha_2\alpha_3)}^{(2)}\right].
    \end{aligned}
\end{equation}
第一行到第二行直接wick 定理展开然后用Eq. \ref{eq:ch4_2point_corr}定义即可,
第二行到第三行用到了如下推导(利用Eq. \ref{eq:ch4_fluc_metric_layer2}到Eq. \ref{eq:ch4_4point_vertex} 
三个公式):
\begin{equation*}
    \begin{aligned}
    & \mathbb{E}\left[\hat{G}_{\alpha_1\alpha_2}^{(2)}\hat{G}_{\alpha_3\alpha_4}^{(2)}\right]=
    \mathbb{E}\left[\left(G_{\alpha_1\alpha_2}^{(2)} + \hat{\Delta G}_{\alpha_1\alpha_2}^{(2)} 
    \right)
    \left(G_{\alpha_3\alpha_4}^{(2)} + \hat{\Delta G}_{\alpha_3\alpha_4}^{(2)} \right)\right] \\
    =& G_{\alpha_1\alpha_2}^{(2)}G_{\alpha_3\alpha_4}^{(2)} +
    G_{\alpha_1\alpha_2}^{(2)}\mathbb{E}\left[\hat{\Delta G}_{\alpha_3\alpha_4}^{(2)}\right] +
    \mathbb{E}\left[\hat{\Delta G}_{\alpha_1\alpha_2}^{(2)}\right]G_{\alpha_3\alpha_4}^{(2)} +
    \mathbb{E}\left[\hat{\Delta G}_{\alpha_1\alpha_2}^{(2)}\hat{\Delta G}_{\alpha_3\alpha_4}^{(2)}
    \right] \\
    =& G_{\alpha_1\alpha_2}^{(2)}G_{\alpha_3\alpha_4}^{(2)} + 
    \mathbb{E}\left[\hat{\Delta G}_{\alpha_1\alpha_2}^{(2)}\hat{\Delta G}_{\alpha_3\alpha_4}^{(2)} 
    \right] = G_{\alpha_1\alpha_2}^{(2)}G_{\alpha_3\alpha_4}^{(2)} + \frac{1}{n_1}
    V_{(\alpha_1\alpha_2)(\alpha_3\alpha_4)}^{(2)}.
    \end{aligned}
\end{equation*}
接下来,根据定义,connected four-point correlators of the second-layer preactivations is
(代入上述求得的$2\&4$-point correlators):
\begin{equation}
    \begin{aligned}
    & \mathbb{E}\left[z_{i_1;\alpha_1}^{(2)}z_{i_2;\alpha_2}^{(2)}
    z_{i_3;\alpha_3}^{(2)}z_{i_4;\alpha_4}^{(2)}\right]\bigg|_{connected} 
    = \mathbb{E}\left[z_{i_1;\alpha_1}^{(2)}z_{i_2;\alpha_2}^{(2)}
    z_{i_3;\alpha_3}^{(2)}z_{i_4;\alpha_4}^{(2)}\right] - \\
    & \left(
        \mathbb{E}\left[z_{i_1;\alpha_1}^{(2)}z_{i_2;\alpha_2}^{(2)}\right]
        \mathbb{E}\left[z_{i_3;\alpha_3}^{(2)}z_{i_4;\alpha_4}^{(2)}\right] +
        \mathbb{E}\left[z_{i_1;\alpha_1}^{(2)}z_{i_3;\alpha_3}^{(2)}\right]
        \mathbb{E}\left[z_{i_2;\alpha_2}^{(2)}z_{i_4;\alpha_4}^{(2)}\right] +
        \mathbb{E}\left[z_{i_1;\alpha_1}^{(2)}z_{i_4;\alpha_4}^{(2)}\right]
        \mathbb{E}\left[z_{i_2;\alpha_2}^{(2)}z_{i_3;\alpha_3}^{(2)}\right]
    \right) \\
    =& \frac{1}{n_1}\left[\delta_{i_1i_2}\delta_{i_3i_4}V_{(\alpha_1\alpha_2)(\alpha_3\alpha_4)}^{(2)}
     +\delta_{i_1i_3}\delta_{i_2i_4}V_{(\alpha_1\alpha_3)(\alpha_2\alpha_4)}^{(2)}
     +\delta_{i_1i_4}\delta_{i_2i_3}V_{(\alpha_1\alpha_4)(\alpha_2\alpha_3)}^{(2)}\right].
    \end{aligned}
\end{equation}
connected 4-point correlators衡量了分布偏离标准高斯分布的程度,因此可以说明\emph{第二层preactivations的分布
$p\left(z^{(2)}\bigg| \mathcal{D}\right)$通常来说是\textbf{non-Gaussian}并且其偏离高斯分布的程度正比于
$\frac{1}{n_1}$和$V_{(\alpha_1\alpha_4)(\alpha_2\alpha_3)}^{(2)}$。也就是说,网络越宽,实际上越接近于高斯分布。}

为了找到对应于上述non-Gaussian$2\&4$-point correlators的action,根据第一章1.2的方法,需要including a quartic 
term in the action. 

\xu{NOTE: 这里各种配coupling项的骚操作,在第一章中有相应理论,
先写个结论,感觉概率的形式没那么重要,暂时先不纠结详细推导了。}
\begin{equation}
    \begin{aligned}
    S[z] &= \frac{1}{2} \sum_{\alpha_1\alpha_2\in\mathcal{D}}g^{\alpha_1\alpha_2}
    \sum_{i=1}^n z_{i;\alpha_1}z_{i;\alpha_2} \\
    &- \frac{1}{8}\sum_{\alpha_1,...,\alpha_4\in\mathcal{D}}v^{(\alpha_1\alpha_2)(\alpha_3\alpha_4)}
    \sum_{i_1,i_2=1}^n z_{i_1;\alpha_1}z_{i_2;\alpha_2}z_{i_3;\alpha_3}z_{i_4;\alpha_4}.
    \end{aligned}
\end{equation}
这里的coupling $v^{(2)}_{(\alpha_1\alpha_2)(\alpha_3\alpha_4)}$与four-point vertex 
$V^{(2)}_{(\alpha_1\alpha_2)(\alpha_3\alpha_4)}$有相同的对称结构并且是对应上的。$g$是跟
$G_{\alpha_1\alpha_2}^{(2)}$对应上的。具体为:
\begin{equation}
    g^{\alpha_1\alpha_2} = G_{(2)}^{\alpha_1\alpha_2} + O(\frac{1}{n_1}),
\end{equation}
\begin{equation}
    v^{(\alpha_1\alpha_2)(\alpha_3\alpha_4)} =
    \frac{1}{n_1}V_{(2)}^{(\alpha_1\alpha_2)(\alpha_3\alpha_4)} + O(\frac{1}{n_1^2}).
\end{equation}
这里上下标颠倒表示为对应矩阵的逆的意思:
\begin{equation}
    V_{(2)}^{(\alpha_1\alpha_2)(\alpha_3\alpha_4)} \equiv \sum_{\beta_1,...,\beta_4}
    G_{(2)}^{\alpha_1\beta_1}G_{(2)}^{\alpha_2\beta_2}G_{(2)}^{\alpha_3\beta_3}
    G_{(2)}^{\alpha_4\beta_4}V_{(\beta_1\beta_2)(\beta_3\beta_4)}^{(2)}.
\end{equation}
这里inverse metric $G_{(2)}^{\alpha_1\alpha_2}$和quartic coupling 
$V_{(2)}^{(\alpha_1\alpha_2)(\alpha_3\alpha_4)}$是input-dependent. 特别地,the effective
strength of interaction (connected 4-point correlator)
between neurons is set by the particular set of inputs to the network. 
\emph{根据公式 Eq. \ref{eq:ch4_2fluc} 以及公式 Eq. \ref{eq:ch4_layer1_metric},可以定性地看出,
$G^{(2)}$和$V^{(2)}$都是与$G^{(1)}$和$1/n_1$正相关,而$G^{(1)}$又是与
$\frac{1}{n_0}\sum_{j=1}^{n_0}x_{j;\alpha_1}x_{j;\alpha_2}$正相关,这就说明,网络
宽度越窄,输入样本之间耦合越强,第二层原子之间的interaction越强。}

\subsection{Deeper Layers: Accumulation of Non-Gaussianity}
拓展到更深层直接按照上一节的公式改成$l$到$l+1$进行迭代即可,相应迭代公式陆续列在如下。

preactivations:
\begin{equation}
    z_{i;\alpha}^{(l+1)} = b_i^{(l+1)} + \sum_{j=1}^{n_l}W_{ij}^{(l+1)}\sigma_{j;\alpha}^{(l)},
    ~~for ~~ i=1,...,n_{l+1}.
\end{equation}
activations:
\begin{equation}
    \sigma_{i;\alpha}^{(l)} \equiv \sigma(z_{i;\alpha}^{(l)}).
\end{equation}
Follow上一节,推导$p\left(z^{(L)}\big| \mathcal{D}\right)$主要有三步:recursion, action,
$1/n$-expansion.

\subsubsection{Recursion}
联合分布:
\begin{equation}
    p\left(z^{(l+1)}, z^{(l)}\big| \mathcal{D}\right) = 
    p\left(z^{(l+1)}\big| z^{(l)}\right)p\left(z^{(l)}\big| \mathcal{D}\right)
\end{equation}
Marginalization over $z^{(l)}$得到$z^{(l+1)}$分布:
\begin{equation}
    p\left(z^{(l+1)}\big| \mathcal{D}\right) = \int \left[\prod_{i,\alpha}dz_{i;\alpha}^{(l)}
    \right]p\left(z^{(l+1)}\big| z^{(l)}\right)p\left(z^{(l)}\big| \mathcal{D}\right).
\end{equation}
$p\left(z^{(l+1)}\big| z^{(l)}\right)$ serves as a \textbf{transition matrix}, bridging 
preactivation distributions in adjacent layers.

Follow 上一节的推导公式,条件概率只需要把$z^{(1)}$换成$z^{(l+1)}$, $x_{j;\alpha}$换成
$\sigma_{j;\alpha}^{(l)}$:
\begin{equation}
    p\left(z^{(l+1)}\big| z^{(l)}\right) = \frac{1}{\sqrt{\big|2\pi \hat{G}^{(l+1)}\big|^{n_{l+1}}}}
    exp\left(-\frac{1}{2}\sum_{i=1}^{n_{l+1}}\sum_{\alpha_1,\alpha_2\in\mathcal{D}}
    \hat{G}_{(l+1)}^{\alpha_1\alpha_2}z_{i;\alpha_1}^{(l+1)}z_{i;\alpha_2}^{(l+1)}\right),
\end{equation}
($l+1$)-th-layer stochastic metric:
\begin{equation}
   \hat{G}_{\alpha_1\alpha_2}^{(l+1)} \equiv C_b^{(l+1)} + C_W^{(l+1)} \frac{1}{n_l}
   \sum_{j=1}^{n_l}\sigma_{j;\alpha_1}^{(l)}\sigma_{j;\alpha_2}^{(l)},
\end{equation}
所有基数阶的correlators都为$0$, 偶数阶的correlators通过Wick's contractions得到,
\begin{equation}
    \mathbb{E}\left[z_{i_1;\alpha_1}^{(l+1)}\cdots z_{i_{2m};\alpha_{2m}}^{(l+1)}\right]
    = \sum_{all~pairings} \delta_{i_{k_1}i_{k_2}}\cdots\delta_{i_{k_{2m-1}}i_{k_{2m}}}
    \mathbb{E}\left[\hat{G}_{\alpha_{k_1}\alpha_{k_2}}^{(l+1)}\cdots
    \hat{G}_{\alpha_{k_{2m-1}}\alpha_{k_{2m}}} \right].
\end{equation}
求和总共有$(2m-1)!!$组不同pair组合($k_1,...,k_{2m}$)构成的项.

The mean of the stochastic metric,
\begin{equation}
    G_{\alpha_1\alpha_2}^{(l+1)} \equiv \mathbb{E}\left[\hat{G}_{\alpha_1\alpha_2}^{(l+1)}
    \right] = C_b^{(l+1)} + C_W^{(l+1)}\frac{1}{n_l}\sum_{j=1}^{n_l}\mathbb{E}\left[
        \sigma_{j;\alpha_1}^{(l)}\sigma_{j;\alpha_2}^{(l)} \right].
\end{equation}
由此可以得到$(l+1)$-th layer的two-point correlator
\begin{equation}
    \mathbb{E}\left[z_{i_1;\alpha_1}^{(l+1)}z_{i_2;\alpha_2}^{(l+1)}\right]
    =\delta_{i_1i_2}\mathbb{E}\left[\hat{G}_{\alpha_1\alpha_2}^{(l+1)}\right]
    =\delta_{i_1i_2}G_{\alpha_1\alpha_2}^{(l+1)}.
    \label{eq:ch4_2point_corr_layer_lp1}
\end{equation}
fluctuation of the mean metric,
\begin{equation}
    \hat{\Delta G}_{\alpha_1\alpha_2}^{(l+1)} \equiv \hat{G}_{\alpha_1\alpha_2}^{(l+1)}
    - G_{\alpha_1\alpha_2}^{(l+1)} = C_W^{(l+1)}\frac{1}{n_l}\sum_{j=1}^{n_l}
    \left(\sigma_{j;\alpha_1}^{(l)}\sigma_{j;\alpha_2}^{(l)}-\mathbb{E}
    \left[\sigma_{j;\alpha_1}^{(l)}\sigma_{j;\alpha_2}^{(l)}\right] \right),
\end{equation}
均值为$0$,
\begin{equation}
    \mathbb{E}\left[\hat{\Delta G}_{\alpha_1\alpha_2}^{(l+1)}\right] = 0,
\end{equation}
幅度为:
\begin{equation}
    \frac{1}{n_l}V_{(\alpha_1\alpha_2)(\alpha_3\alpha_4)}^{(l+1)} \equiv
    \mathbb{E}\left[\hat{\Delta G}_{\alpha_1\alpha_2}^{(l+1)}
    \hat{\Delta G}_{\alpha_3\alpha_4}^{(l+1)}\right] = \mathbb{E}\left[
    \hat{G}_{\alpha_1\alpha_2}^{(l+1)}\hat{G}_{\alpha_3\alpha_4}^{(l+1)} \right]
    -G_{\alpha_1\alpha_2}^{(l+1)}G_{\alpha_3\alpha_4}^{(l+1)}.
\end{equation}
connected four-point correlator:
\begin{equation}
    \begin{aligned}
    & \mathbb{E}\left[z_{i_1;\alpha_1}^{(l+1)}z_{i_2;\alpha_2}^{(l+1)}
        z_{i_3;\alpha_3}^{(l+1)}z_{i_4;\alpha_4}^{(l+1)} \right]_{connected} \\
    =& \frac{1}{n_l}\left[\delta_{i_1i_2}\delta_{i_3i_4}V_{(\alpha_1\alpha_2)
    (\alpha_3\alpha_4)}^{(l+1)} + \delta_{i_1i_3}\delta_{i_2i_4}V_{(\alpha_1\alpha_3)
    (\alpha_2\alpha_4)}^{(l+1)} + \delta_{i_1i_4}\delta_{i_2i_3}V_{(\alpha_1\alpha_4)
    (\alpha_2\alpha_3)}^{(l+1)} \right].
    \end{aligned}
    \label{eq:ch4_conn_4point_corr_layer_l}
\end{equation}
接下来就是通过action求解得到$p\left(z^{(l+1)}\big| \mathcal{D}\right)$.

\subsubsection{Action}
preactivation distribution $p\left(z^{(l)}\big| \mathcal{D}\right)$,
\begin{equation}
    p\left(z^{(l)}\big| \mathcal{D}\right) = \frac{e^{-S(z^{(l)})}}{Z(l)},
\end{equation}
$l$-th layer partition function 
\begin{equation}
    Z(l) \equiv \int \left[\prod_{i,\alpha}dz_{i;\alpha}^{(l)}\right]e^{-S(z^{(l)})},
\end{equation}
根据上一节,action表达为:
\begin{equation}
    \begin{aligned}
    S(z^{(l)}) &\equiv \frac{1}{2}\sum_{i=1}^{n_l}\sum_{\alpha_1,\alpha_2\in \mathcal{D}}
    g_{(l)}^{\alpha_1\alpha_2}z_{i;\alpha_1}^{(l)}z_{i;\alpha_2}^{(l)} \\
    &- \frac{1}{8}\sum_{i_1,i_2=1}^{n_l}\sum_{\alpha_1,...,\alpha_4\in \mathcal{D}}
    v_{(l)}^{(\alpha_1\alpha_2)(\alpha_3\alpha_4)}z_{i_1;\alpha_1}^{(l)}z_{i_2;\alpha_2}^{(l)}
    z_{i_3;\alpha_3}^{(l)}z_{i_4;\alpha_4}^{(l)} + ...
    \end{aligned}
\end{equation}
与上一节的推导一样,$2$阶和$4$阶couplings用$2$ 阶和$4$ 阶correlators计算如下,
\begin{equation}
    \begin{aligned}
    g_{(l)}^{(\alpha_1\alpha_2)} = &~ G_{(l)}^{\alpha_1\alpha_2} + O(v,...), \\
    v_{(l)}^{(\alpha_1\alpha_2)(\alpha_3\alpha_4)} = &~ \frac{1}{n_{l-1}}
    V_{(l)}^{(\alpha_1\alpha_2)(\alpha_3\alpha_4)} + O(v^2,...), \\
    V_{(l)}^{(\alpha_1\alpha_2)(\alpha_3\alpha_4)} \equiv & \sum_{\beta_1,...,\beta_4\in 
    \mathcal{D}} G_{(l)}^{\alpha_1\beta_1}G_{(l)}^{\alpha_2\beta_2}
    G_{(l)}^{\alpha_3\beta_3}G_{(l)}^{\alpha_4\beta_4}
    V_{(\beta_1\beta_2)(\beta_3\beta_4)}^{(l)}.
    \end{aligned}
    \label{eq:ch4_coupling_layer_l}
\end{equation}
上述式子中的高阶项可以被忽略if and only if the quartic coupling 
$v$ and higher-order couplings are perturbatively small (当网络足够宽的时候).

\subsubsection{Large-width expansion}
按照之前讨论,各层原子数比较大的时候,高阶项$O(\frac{1}{n_l})$都会比较小,其对分布带来的扰动都可以合理
忽略,各种计算都可以被适当简化,i.e. this large-but-finite-width regime is where networks 
become both practically usable and theoretically tractable.
\begin{equation}
    n_1,n_2,...,n_{L-1} \sim n \gg 1.
\end{equation}
Mean metric由于都是确定量而且与$n$无关,加上前两层的情况,可以设定 $V^{(l)}=O(1)$, $G^{(l)}=O(1)$
对各层均满足。根据Eq. \ref{eq:ch4_coupling_layer_l} 中correlators和couplings的关系可以得知,在
$n\gg 1$ setting下,$v_{(l)}=O(1/n)$ is perturbatively small at the $l$-th layer and that 
the quadratic coupling is given by $g_{(l)}=G_{(l)}+O(1/n)$. In carrying out this inductive 
proof, we obtain the recursion relations that govern the change in the preactivation 
distributions from the $l$-th layer to the ($l + 1$)-th layer.

two-point correlator in the ($l+1$)-th layer 可以进一步细化为
\begin{equation}
    G_{\alpha_1\alpha_2}^{(l+1)} = C_b^{(l+1)} + C_W^{(l+1)}\frac{1}{n_l}
    \sum_{j=1}^{n_l}\mathbb{E}\left[\sigma_{j;\alpha_1}^{(l)}\sigma_{j;\alpha_2}^{(l)}
    \right]
\end{equation}
插入上述coupling $g_{(l)} = G_{(l)} + O(1/n)$ 以及 $v_{(l)} = O(1/n)$, 得到
\begin{equation}
    \begin{aligned}
    G_{\alpha_1\alpha_2}^{(l+1)} = C_b^{(l+1)} + C_W^{(l+1)}\langle \sigma_{\alpha_1}
    \sigma_{\alpha_2}\rangle_{G^{(l)}} + O\left(\frac{1}{n}\right),
    \end{aligned}
    \label{eq:ch4_2point_corr_layer_l}
\end{equation}
\xu{NOTE: 补充细节推导.}

接下来推导four-point correlator, 
\begin{equation}
    \begin{aligned}
        & \frac{1}{n_l}V_{(\alpha_1\alpha_2)(\alpha_3\alpha_4)}^{(l+1)} \\
        =& \left(\frac{C_W^{(l+1)}}{n_l}\right)^2\sum_{j,k=1}^{n_l}\left\{
            \mathbb{E}\left[\sigma_{j;\alpha_1}^{(l)}\sigma_{j;\alpha_2}^{(l)}
            \sigma_{k;\alpha_3}^{(l)}\sigma_{k;\alpha_4}^{(l)} \right]
            -\mathbb{E}\left[\sigma_{j;\alpha_1}^{(l)}\sigma_{j;\alpha_2}^{(l)}\right] 
            \mathbb{E}\left[\sigma_{k;\alpha_1}^{(l)}\sigma_{k;\alpha_2}^{(l)}\right]
            \right\}
    \end{aligned}
    \label{eq:ch4_4point_corr_layer_l}
\end{equation}
$j=k$ 的情况下,为同一个原子的correlation,
\begin{equation}
    \begin{aligned}
    &\mathbb{E}\left[\sigma_{j;\alpha_1}^{(l)}\sigma_{j;\alpha_2}^{(l)}
    \sigma_{j;\alpha_3}^{(l)}\sigma_{j;\alpha_4}^{(l)} \right]
    - \mathbb{E}\left[\sigma_{j;\alpha_1}^{(l)}\sigma_{j;\alpha_2}^{(l)}\right] 
    \mathbb{E}\left[\sigma_{j;\alpha_3}^{(l)}\sigma_{j;\alpha_4}^{(l)}\right] \\
    =& \langle\sigma_{\alpha_1}\sigma_{\alpha_2}\sigma_{\alpha_3}\sigma_{\alpha_4}
    \rangle_{G^{(l)}} - \langle\sigma_{\alpha_1}\sigma_{\alpha_2}\rangle_{G^{(l)}}
    \langle\sigma_{\alpha_3}\sigma_{\alpha_4}\rangle_{G^{(l)}} + 
    O\left(\frac{1}{n}\right).
    \end{aligned}
\end{equation}
$j\neq k$的情况下,为两个原子之间的correlation,
\begin{equation}
    \begin{aligned}
    &\mathbb{E}\left[\sigma_{j;\alpha_1}^{(l)}\sigma_{j;\alpha_2}^{(l)}
    \sigma_{k;\alpha_3}^{(l)}\sigma_{k;\alpha_4}^{(l)} \right] - 
    \mathbb{E}\left[\sigma_{j;\alpha_1}^{(l)}\sigma_{j;\alpha_2}^{(l)}\right]
    \mathbb{E}\left[\sigma_{k;\alpha_1}^{(l)}\sigma_{k;\alpha_2}^{(l)}\right] \\
    =& \frac{1}{4n_{l-1}}\sum_{\beta_1,...,\beta_4\in \mathcal{D}} 
    V_{(l)}^{(\beta_1\beta_2)(\beta_3\beta_4)}\langle\sigma_{\alpha_1}\sigma_{\alpha_2}
    (z_{\beta_1}z_{\beta_2} - g_{\beta_1\beta_2})\rangle_{G^{(l)}}
    \langle\sigma_{\alpha_3}\sigma_{\alpha_4} (z_{\beta_3}z_{\beta_4} - 
    g_{\beta_3\beta_4})\rangle_{G^{(l)}} \\
    & + O\left(\frac{1}{n^2}\right).
    \end{aligned}
\end{equation}
\xu{NOTE: 上面两个公式在之前章节有推导,比较复杂,暂时就先用了,后续有需要再补上...}

把上述两式代入 Eq. \ref{eq:ch4_4point_corr_layer_l} 中对$\sum$进行展开,得到
\begin{equation}
    \begin{aligned}
    & \frac{1}{n_l}V_{(\alpha_1\alpha_2)(\alpha_3\alpha_4)} \\ 
    =& \frac{1}{n_l}\left(C_W^{(l+1)}\right)^2[\langle\sigma_{\alpha_1}\sigma_{\alpha_2}
    \sigma_{\alpha_3}\sigma_{\alpha_4}\rangle_{G^{(l)}} - 
    \langle\sigma_{\alpha_1}\sigma_{\alpha_2}\rangle_{G^{(l)}}
    \langle\sigma_{\alpha_3}\sigma_{\alpha_4}\rangle_{G^{(l)}}] \\
    & + \frac{1}{n_{l-1}}\frac{\left(C_W^{(l+1)}\right)^2}{4} \sum_{\beta_1,...,
    \beta_4\in \mathcal{D}} V_{(l)}^{(\beta_1\beta_2)(\beta_3\beta_4)}\langle
    \sigma_{\alpha_1}\sigma_{\alpha_2}(z_{\beta_1}z_{\beta_2} - 
    g_{\beta_1\beta_2})\rangle_{G^{(l)}} \\
    & \times \langle\sigma_{\alpha_3}\sigma_{\alpha_4}(z_{\beta_3}z_{\beta_4} - 
    g_{\beta_3\beta_4})\rangle_{G^{(l)}} + O\left(\frac{1}{n^2}\right).
    \end{aligned}
\end{equation}
可以看出比较重要的一点,
\begin{equation}
    \frac{1}{n_l}V^{(l+1)} = O\left(\frac{1}{n}\right).
\end{equation}
代入到 Eq. \ref{eq:ch4_conn_4point_corr_layer_l} 中,$\mathbb{E}\left[z_{i_1;\alpha_1}^{(l+1)}
z_{i_2;\alpha_2}^{(l+1)}z_{i_3;\alpha_3}^{(l+1)}z_{i_4;\alpha_4}^{(l+1)} \right]_{connected} 
= \frac{1}{n_l} [\delta_{i_1i_2}\delta_{i_3i_4}V_{(\alpha_1\alpha_2)(\alpha_3\alpha_4)}^{(l+1)} 
+ \delta_{i_1i_3}\delta_{i_2i_4}V_{(\alpha_1\alpha_3)(\alpha_2\alpha_4)}^{(l+1)} + 
\delta_{i_1i_4}\delta_{i_2i_3}V_{(\alpha_1\alpha_4)(\alpha_2\alpha_3)}^{(l+1)}]
= O\left(\frac{1}{n}\right)$.
类似的,recursion for quartic coupling 可以直接上式中$G$换成$g$, $V$换成$v$并且忽略$1/n^2$项即可。

\subsection{Marginalization Rules}
\xu{NOTE: 公式实在太多,这一章不影响大的理解,暂时先不搞了,主要是讲
$p\left(z^{(l+1)}\big| \mathcal{D}\right)$ 来源于对
$p\left(z^{(l+1)},z^{(l)}\big| \mathcal{D}\right) = 
p\left(z^{(l+1)}\big| z^{(l)}\right)p\left(z^{(l)}\big| \mathcal{D}\right)$的积分
的推导规则, 貌似在最后一节summize的时候这个marginalization还是一个比较关键的概念,
后续再补上。}

\subsection{RG Flow and RG Flow}
\xu{NOTE: 这一节的解释太漂亮了,一个字我都不好意思翻译,几乎是全文保留}.

The goal of this chapter was to find the marginal distribution of preactivations
$p(z^{(l)}\big| \mathcal{D})$ in a given layer $l$ in terms of an \textbf{effective action}
with data-dependent couplings. These couplings change - or \textbf{run} - from layer to 
layer, and the running is determined via recursions, which in turn determine 
how the distribution of preactivations changes with depth. Equivalently, 
these recursions tell us how correlation functions of preactivations evolve with layer. 
In this language, starting with independent neurons in the first layer (§4.1), we saw 
how interactions among neurons are induced in the second layer (§4.2) and 
then amplified in deeper layers (§4.3).

Concretely, let's summarize the behavior of finite-width networks to leading order
in the wide-network expansion. Expressing the two-point correlator of 
preactivations in terms of the \textbf{kernel} $K_{\alpha_1\alpha_2}^{(l)}$ as
\begin{equation}
    \mathbb{E}\left[z_{i_1;\alpha_1}^{(l)}\right] = \delta_{i_1i_2}G_{\alpha_1\alpha_2}^{(l)}
    = \delta_{i_1i_2}\left[K_{\alpha_1\alpha_2}^{(l)} + O\left(\frac{1}{n}\right)\right].
\end{equation}
这里的$K_{\alpha_1\alpha_2}^{(l)}$相当于公式 Eq. \ref{eq:ch4_2point_corr_layer_l} 中的
$G^{(l-1)}$, 按上一节讨论,是一个$O(1)$的量.

Expressing the four-point connected correlator in terms of the \textbf{four-point vertex}
$V_{(\alpha_1\alpha_2)(\alpha_3\alpha_4)}^{(l)}$ as
\begin{equation}
    \begin{aligned}
    & \mathbb{E}\left[z_{i_1;\alpha_1}^{(l)}z_{i_2;\alpha_2}^{(l)}
        z_{i_3;\alpha_3}^{(l)}z_{i_4;\alpha_4}^{(l)} \right]_{connected} \\
    =& \frac{1}{n_{l-1}}\left[\delta_{i_1i_2}\delta_{i_3i_4}V_{(\alpha_1\alpha_2)
    (\alpha_3\alpha_4)}^{(l)} + \delta_{i_1i_3}\delta_{i_2i_4}V_{(\alpha_1\alpha_3)
    (\alpha_2\alpha_4)}^{(l)} + \delta_{i_1i_4}\delta_{i_2i_3}V_{(\alpha_1\alpha_4)
    (\alpha_2\alpha_3)}^{(l)} \right].
    \end{aligned}
    \label{eq:ch4_recur_4point_corr}
\end{equation}
The running of these correlators is given by the recursions
\begin{equation}
    \begin{aligned}
    K_{\alpha_1\alpha_2}^{(l+1)} =&~C_b^{(l+1)} + C_W^{(l+1)}\langle \sigma_{\alpha_1}
    \sigma_{\alpha_2}\rangle_{K^{(l)}} \\
    V_{(\alpha_1\alpha_2)(\alpha_3\alpha_4)}^{(l+1)}  
    =& \left(C_W^{(l+1)}\right)^2[\langle\sigma_{\alpha_1}\sigma_{\alpha_2}
    \sigma_{\alpha_3}\sigma_{\alpha_4}\rangle_{K^{(l)}} - 
    \langle\sigma_{\alpha_1}\sigma_{\alpha_2}\rangle_{K^{(l)}}
    \langle\sigma_{\alpha_3}\sigma_{\alpha_4}\rangle_{K^{(l)}}] \\
    & + \frac{1}{4}\frac{n_l}{n_{l-1}}\left(C_W^{(l+1)}\right)^2 \sum_{\beta_1,...,
    \beta_4\in \mathcal{D}} V_{(l)}^{(\beta_1\beta_2)(\beta_3\beta_4)}\langle
    \sigma_{\alpha_1}\sigma_{\alpha_2}(z_{\beta_1}z_{\beta_2} - 
    K_{\beta_1\beta_2}^{(l)})\rangle_{K^{(l)}} \\
    & \times \langle\sigma_{\alpha_3}\sigma_{\alpha_4}(z_{\beta_3}z_{\beta_4} - 
    K_{\beta_3\beta_4}^{(l)})\rangle_{K^{(l)}} + O\left(\frac{1}{n^2}\right).
    \end{aligned}
    \label{eq:ch4_recur_all_corr}
\end{equation}
\emph{这里可以看出$V^{(l)}$是逐层累加的 (上述第二个等式的前两项为正), 也就是说$\mathbb{E}\left[z_{i_1;\alpha_1}^{(l)}z_{i_2;\alpha_2}^{(l)}
z_{i_3;\alpha_3}^{(l)}z_{i_4;\alpha_4}^{(l)} \right]_{connected}$从$1$到$l$层逐渐增大,
i.e. 深度越深,$p(z^{(l)}|\mathcal{D})$ 越偏离标准高斯分布,$z^{(l)}$各原子间interaction越强}.

The indices on the four-point vertex are raised by the inverse metric $G_{(l)}$
\begin{equation}
    \begin{aligned}
    V_{(l)}^{(\alpha_1\alpha_2)(\alpha_3\alpha_4)} \equiv & \sum_{\beta_1,...,\beta_4\in 
    \mathcal{D}} G_{(l)}^{\alpha_1\beta_1}G_{(l)}^{\alpha_2\beta_2}
    G_{(l)}^{\alpha_3\beta_3}G_{(l)}^{\alpha_4\beta_4}
    V_{(\beta_1\beta_2)(\beta_3\beta_4)}^{(l)} \\
    =& \sum_{\beta_1,...,\beta_4\in 
    \mathcal{D}} K_{(l)}^{\alpha_1\beta_1}K_{(l)}^{\alpha_2\beta_2}
    K_{(l)}^{\alpha_3\beta_3}K_{(l)}^{\alpha_4\beta_4}
    V_{(\beta_1\beta_2)(\beta_3\beta_4)}^{(l)} + O\left(\frac{1}{n}\right).
    \end{aligned}
\end{equation}

This flow is very reminiscent of the following heuristic picture, which is 
offered as an explanation for how neural networks are supposed to work: 
given an input, such as the image of a cat, the first few layers identify 
low-level features from the pixels - such as the \textbf{edges} between areas of low 
and high intensity - and then the middle layers assemble these low-level 
features into mid-level features - such as the texture and pattern of \textbf{fur} - 
which are further aggregated in deeper layers into higher-level representations 
- such as \textbf{tails} and \textbf{ears} - which the last layer combines into 
an estimate of the probability that original pixels represents a \textbf{cat}.
Indeed, some studies support this hierarchically-ordered arrangement of feature 
representation in trained networks (\emph{这里引用了Fergus那篇visualizing cnn的论文,很有意思}). 
The desirability of such an arrangement emphasizes both the role and importance 
of depth in deep learning.

Some of the terms we used in discussing this heuristic picture can actually be 
given more precise definitions. For instance, each neuron in the network - 
including not only those in the output layer but also those in the hidden layers - 
is a scalar function of the input and called a \textbf{feature}. The neurons of a given layer 
can be organized into a vector-valued function of the input, which we'll refer to as 
a \textbf{representation}. In terms of these concepts, our formalism tracks the 
transformation of representations from one layer to the next. It is this flow of 
representations that we term \textbf{representation group flow} or \textbf{RG flow} 
for short. RG flow is induced via the repeated 
marginalization ($p\left(z^{(l+1)}\big| \mathcal{D}\right)$ 来源于
联合分布$p\left(z^{(l+1)},z^{(l)}\big| \mathcal{D}\right) = 
p\left(z^{(l+1)}\big| z^{(l)}\right)p\left(z^{(l)}\big| \mathcal{D}\right)$对$z^{(l)}$
的积分, $p\left(z^{(l)}\big| \mathcal{D}\right)$就是上一层fine-grained features的分布)
of fine-grained features in the shallow layers to give a coarse-grained 
representation in the output layer. Our notion of RG flow makes the heuristic picture 
given above concrete.

This pattern of coarse-graining has a parallel in theoretical physics, known as 
\textbf{renormalization group flow} or \textbf{RG flow} for short. In this case, 
the RG flow is generated by the repeated marginalization of the microscopic fine-grained 
degrees of freedom in the system in order to obtain an effective theory of the system 
in terms of macroscopic coarse-grained variables. Analogously, the physical couplings
(\emph{貌似和前面几节里面coupling $v\&g$的概念有点对上了}) 
controlling the interactions of these effective degrees of freedom run with the 
length scale at which they are probed - e.g. the effective charge of the electron 
will change when interrogated at different scales (\xu{NOTE: 这个物理现象不太懂,求解释}). 
Similar to the recursion equations 
describing the running couplings of the network representations, one can derive 
differential equations - historically called beta functions - that govern the running 
of the physical couplings with scale.

In this scenario, the degrees of freedom are represented by a field $\phi(x)$ 
that may take different values as a function of spacetime coordinate $x$. 
First, one divides $\phi(x)$ into fine-grained variables $\phi^+$ consisting of 
high-frequency modes and coarse-grained variables $\phi^-$ consisting of 
low-frequency modes, such that the field decomposes as $\phi(x) = \phi^+(x) + \phi^-(x)$. 

The full distribution is governed by the full action (\emph{这里貌似和上面几节里面根据
$2\&4$-point correlator 求分布的action $S$有点对上了})
\begin{equation}
    S_{full}(\phi) = S(\phi^+) + S(\phi^-) + S_I(\phi^+, \phi^-).
\end{equation}
where in particular the last term describes the interactions between these two sets 
of modes.

Now, if all we care about are observables that depend only on the 
coarse-grained modes $\phi^-$ at macroscopic scales - and such long-range 
scales are usually the relevant ones for experiments - then this full description 
is too cumbersome to usefully describe the outcome of such experiments. 
In order to obtain an effective description in terms of only these coarse-grained variable 
$\phi^-$, we can integrate out (i.e. marginalizes over) the fine-grained variables 
$\phi^+$ as 
\begin{equation}
    e^{-S_{eff}(\phi^-)} = \int d\phi^+ e^{-S_{full}(\phi)}.
    \label{eq:ch4_int_eff}
\end{equation}
and obtain an \textbf{effective action} $S_{eff}(\phi^−)$, providing an effective theory 
for the observables of experimental interest. In practice, this marginalization is carried 
out scale by scale, dividing up the field as $\phi = \phi^{(1)} + . . . + \phi^{(L)}$ 
from microscopic modes $\phi^{(1)}$ all the way to macroscopic modes $\phi^{(L)} = \phi^−$,
 and then integrating out the variables $\phi^{(1)}, . . . , \phi^{(L−1)}$ in sequence. 
 Tracking the flow of couplings in the effective action through this marginalization 
 results in the aforementioned beta functions, and in solving these differential 
 equations up to the scale of interest, we get an effective description of 
 observables at that scale. 

This is precisely what we have been doing in this chapter for neural networks. 
The full field $\phi$ is analogous to a collection of all the preactivations 
$\{z^{(1)},…,z^{(L)}\}$.  Their distribution is governed by the full joint 
distribution of preactivations 
\begin{equation}
    p\left(z^{(1)},...,z^{(L)}\big| \mathcal{D}\right) = p\left(z^{(L)}\big| z^{(L-1)}\right)
    \cdots p\left(z^{(2)}\big|z^{(1)}\right)p\left(z^{(1)}\big| \mathcal{D}\right),
\end{equation}
with the full action
\begin{equation}
    S_{full}\left(z^{(1)},...,z^{(L)}\right) \equiv \sum_{l=1}^L S_M\left(z^{(l)}\right)
    + \sum_{l=1}^{L-1}S_I\left(z^{(l+1)}\big| z^{(l)}\right).
\end{equation}
Here, the full action is decomposed into the mean quadratic action for variables $z^{(l)}$
(对应于$2$-point correlator)
\begin{equation}
    S_M\left(z^{(l)}\right) = \frac{1}{2}\sum_{i=1}^{n_l}\sum_{\alpha_1,\alpha_2\in\mathcal{D}}
    G_{(l)}^{\alpha_1\alpha_2}z_{i;\alpha_1}^{(l)}z_{i;\alpha_2}^{(l)}
\end{equation}
in terms of the mean metric $G^{(l)}$, and the interaction between neighboring layers
(对应于connected $2$-point correlator)
\begin{equation}
    S_I\left(z^{(l+1)}\big| z^{(l)}\right)=\frac{1}{2}\sum_{i=1}^{n_{l+1}}
    \sum_{\alpha_1,\alpha_2\in\mathcal{D}}\left[
    \hat{G}_{(l+1)}^{\alpha_1\alpha_2}\left(z^{(l)}\right) - G_{(l+1)}^{\alpha_1\alpha_2}\right]
    z_{i;\alpha_1}^{(l+1)} z_{i;\alpha_2}^{(l+1)}.
 \end{equation}

 Here we emphasized that the stochastic metric $\hat{G}_{\alpha_1\alpha_2}^{(l+1)}$ 
 is a function of $z^{(l)}$, and the induced coupling of $z^{(l)}$ with $z^{(l+1)}$ 
 is what leads to the interlayer interactions.

Now, if all we care about are observables that depend only on the outputs of the network
- then this full description 
is too cumbersome. In order to obtain an effective (i.e. useful) description of 
the distribution of outputs $z^{(L)}$, we can marginalizes over all the features 
$\{z^{(1)},...,z^{(L-1)}\}$ as
\begin{equation}
    e^{-S_{eff}\left(z^{(L)}\right)} = \int \left[\prod_{l=1}^{L-1}dz^{(l)}\right]
    e^{-S_{full}\left(z^{(1)},...,z^{(L)}\right)}.
\end{equation}
just as we integrated out the fine-grained modes $\phi^+$ in Eq. \ref{eq:ch4_int_eff} to 
get the effective description in terms of coarse-grained modes $\phi^−$. 
And, just like in the field theory example, rather than carrying out this 
marginalization all at once, we proceeded sequentially, integrating out 
the preactivations layer by layer. This resulted in the recursion relations 
Eq. \ref{eq:ch4_recur_all_corr}, and in solving these recursion relations 
up to the depth of interest, we get an effective description of 
neural network output at that depth. 

Given this perspective, our RG flow can address the effect of 
the \emph{deep} in \emph{deep learning}. For instance, as a network get 
deeper, do the interactions between neurons - encoded in the finite-width 
corrections ($O(1/n)$项无法忽略) such as the four-point vertex $V^{(l)}$ - 
get amplified or attenuated? (\xu{NOTE: 到底interaction大好还是小好?})  
In the language of RG flow, couplings that grow with the flow are called 
\textbf{relevant} and those that shrink are called \textbf{irrelevant}. 

This question has important implications for deep learning. If all the 
finite-width couplings were irrelevant, then finite-width networks 
would asymptote to infinite-width architectures under RG flow. 越irrelevant 
表示interaction between neurons越小,各层neuron的分布越接近于标准的高斯分布,那么
deep也就失去了价值。Fortunately we'll soon find that the couplings are relevant, 
making our life richer, albeit more complicated. In the next chapter, we'll
show that finite networks deviate more and more from their infinite-width 
counterparts as they get deeper. This has important practical consequences 
in controlling the instantiation-to-instantiation fluctuations in supervised 
training (§$10$) as well as allowing networks to learn nontrivial representations 
of their input (§$11$). 

\xu{NOTE: 所以传说中的distributed representation in deep learning,
希望deep learning models 各个维度的features解耦/相互独立反而是不好的?保留越来越强的interaction
反而是好的?}

